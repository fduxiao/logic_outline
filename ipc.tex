\chapter{Intuitionistic Logic}
\label{intuitionistic-logic}

In this chapter I mainly discuss the intuitionistic propositional logic
(or intuitionistic propositional logic calculus, IPC) contrast to the
classical propositional calculus\footnote{\autoref{classical-logic}} 
(CPC), which is the logic used
in Curry-Howard correspondence. In this logic, the {\it Law of Excluded
Middle} (LEM) is not included as a tautology, but it also has the 
completeness part with the semantic model of Heyting algebra. 
This doesnot mean the logic is not as expressive as classical logic.
Indeed, Glivenko gave a proof in 1929 \cite{glivenko1929quelques} saying
that each CPC tautology can be proved in IPC with a {\it double negation}
translation. The key fact is the elimination of double negation, i.e.,
$\neg\neg\neg\neg p\proves \neg\neg p$. (Actually we can achieve a better
$\neg\neg\neg p\proves\neg p$.) This further gives rise to the {\it
Continuation Passing Style} (CPS) transformation in programming and 
combinatory logic, and it can be described with a monad, which is a
magic tool to represent some logic calculus or {\it encapsulation}.

\section{Natural Deduction}
\newcommand{\formula}{\mathscr{F}}

Given a set $V$ of propositional variables, the well formed
formulas are defined as the recursive set:
$$
    \formula ::= \bot 
        \mid V 
        \mid (\formula \to \formula) 
        \mid (\formula \vee \formula)
        \mid (\formula \wedge \formula)
$$

The variables $V$ represent some particular propositions. For those in
$V$, I will use capital English letters $A,B,C$, but for a member in 
$\formula$, I will use lowercase letters $a,b,c$ or Greek letters 
$\varphi,\psi,\theta$ unless I mention otherwise explicitly. 
The arrow $\to$ means implication, the wedge $\wedge$
means conjunctions, i.e. $a\wedge b$ means `$a$ and $b$', and
$a\vee b$ means disjunction `$a$ or $b$'. The bottom $\bot$ means 
absurdity and thus the negation $\neg a$ is an alias to $a\to\bot$.
The negation only means $a$ will trigger the absurdity and thus everything
according to the principle of explosion (ex falso quodlibet). Together
with other interpretation of formulas, we form a {\it natural deduction
system}. 

\begin{definition}
    Let $\Gamma$ be a set of formulas, for any formula $\varphi$, we define
    the relation $\Gamma\proves\varphi$ with the following rules:
    \begin{itemize}
        \item Axiom: $\Gamma,\varphi\proves\varphi$
        \item $\wedge$-Introduction: $\infer{\Gamma\proves\varphi\wedge\psi}
            {\Gamma\proves\varphi & \Gamma\proves \psi}$

            $\wedge$-Elimination:
            $\infer{\Gamma\proves\varphi}{\Gamma\proves\varphi\wedge\psi}$\ \  
            $\infer{\Gamma\proves\psi}{\Gamma\proves\varphi\wedge\psi}$
        \item $\vee$-Introduction: 
            $\infer{\Gamma\proves\varphi\vee\psi}{\Gamma\proves\varphi}$\ \  
            $\infer{\Gamma\proves\varphi\vee\psi}{\Gamma\proves\psi}$

            $\vee$-Elimination:
            $\infer{\Gamma\proves\rho}
            {\Gamma,\varphi\proves\rho & \Gamma,\psi\proves\rho & \Gamma\proves\varphi\vee\psi}$

        \item $\to$-Introduction: 
            $\infer{\Gamma\proves\varphi\to\psi}{\Gamma,\varphi\proves\psi}$

            $\to$-Elimination:
            $\infer{\Gamma\proves\psi}{\Gamma\proves\varphi\to\psi & \Gamma\proves\varphi}$
        \item $\bot$-Elimination:
            $\infer{\Gamma\proves\varphi}{\Gamma\proves\bot}$
    \end{itemize}
\end{definition}

Unlike CPC, we define the $\proves$
relation with introduction and elimination rules instead of logical axioms
this time. The horizonal line means if you have obtained the relation 
$\proves$ above the line, then you can derive the relation under the line. 
Thus a proof of $\Gamma\proves\varphi$ is a tree, e.g. (note $\neg p$ is an
abbreviation for $p\to\bot$)

\begin{equation}
    \label{nnLEM}
    $$
    \infer{\proves\neg\neg(p\vee\neg p)}{
        \infer{\neg(p\vee\neg p)\proves\bot}{
            \neg(p\vee\neg p)\proves \neg(p\vee\neg p)
            &
            \infer{\neg(p\vee\neg p)\proves p\vee\neg p}{
                \infer{\neg(p\vee\neg p)\proves\neg p}{
                    \infer{\neg(p\vee\neg p),p\proves\bot} {
                        \neg(p\vee \neg p), p\proves\neg(p\vee\neg p)
                        &
                        \infer{\neg(p\vee\neg p), p\proves p\vee\neg p}{
                            \neg(p\vee\neg p),p\proves p
                        }
                    }
                }
            }
        }
    }
    $$
\end{equation}

By induction on the tree, we can also prove the following.
$$
\infer{\Gamma,q\proves p}{\Gamma\proves p}
$$

The natural deduction, I think, is a generalization of {\it deduction theorem}
(\autoref{deduction-theorem}). There's actually a sequential form of natural
deduction system, which is handled with {\it combinatory logic} (Chapter 5
of \cite{Curry-Howard}), but unlike the sequential version, the deduction
rule is more intuitive in our proof. (Think about how Curry-Howard
correspondence helps to quickly check a formula 
$(a\to b)\to (b\to c)\to(a\to c)$ is a tautology or not.)

\section{Heyting Algebra}
Obviously you can still define the assignment $v:V\to\two$ and extend
it to $\formula$. No doubt you can find the `soundness' part, i.e.,
if $v(\varphi)=\one$ for all $\varphi\in\Gamma$, and $\Gamma\proves\psi$,
then $v(\psi)=\one$. Let me make this more general.
\begin{definition}\footnote{A definition from \cite{Curry-Howard}.}
    A field of sets (over $X$) is a nonempty family $\mathcal{R}$ of
    subsets of $X$ closed under unions, intersections and complements. 
\end{definition}
Note that since $\mathcal{R}$ is closed under complements ($-$), unions 
and intersections, for any $A\in\mathcal{R}$, $X=A\cup(-A)\in\mathcal{R}$
and $\emptyset=A\cap(-A)\in\mathcal{R}$.

We can define the assignment as a function $v: V\to\mathcal{R}$ with
extension $\bar{v}:\formula\to\mathcal{R}$:
\begin{itemize}
    \item $\bar{v}(A)=v(A)$ for all $A\in V$.
    \item $\bar{v}(\bot)=\emptyset$.
    \item $\bar{v}(a\vee b)=\bar{v}(a)\cup \bar{v}(b)$.
    \item $\bar{v}(a\wedge b)=\bar{v}(a)\cap \bar{v}(b)$.
    \item $\bar{v}(a\to b)=(X-\bar{v}(a))\cup \bar{v}(b)$.
\end{itemize}

(Think about the set defined by a formula in model theory. What if you
take the conjunction or disjunction of two such formulas?).

The behavior of field of sets can be generalzed by {\it Boolean algebra}.
In \cite{Curry-Howard}, it is define as an algebra system 
$\mathcal{B}=\angletuple{B,\cup,\cap,-,0,1}$ such that
\begin{enumerate}
    \item $\cup,\cap$ are associative and commutative and mutually distributive
    \item $a\cup 0=a$, $a\cap 1=a$
    \item $-a\cup a=1$, $-a\cap a=0$
\end{enumerate}
\newcommand{\Imply}{\Rightarrow}
We can then interpret truth as $v(p)=1$ for some Boolean algebra.
Clearly we cannot have the completeness with such an interpretation,
because $v(p\vee\neg p)=1$ for any Boolean algebra, but it is not provable.
\cite{Curry-Howard} further gives the model of {\it Heyting algebra} as
$\mathcal{H}=\angletuple{H,\cup,\cap,\Imply,-,0,1}$ with some extra
axioms such as distributivity, but I prefer the category-theoretical
version in \cite{cat-awodey}. (Recall the relation between lambda
calculus and Cartesian Closed Category). 

\begin{definition}
    A Heyting algebra is a poset $\le$ (category) with finite
    products(meets), coproducts(joins) and exponentials.
\end{definition}
\begin{remark}
    In a poset, the product $p\times q$ is acturally the 
    {\it greatest lower bound} $p\cap q$ (or $p\wedge q$), with the coproduct
    $p+q$ the {\it least upper bound} $p\cup q$ (or $p\vee q$).
    The nullary product and coproduct are the terminal object $1$
    and initial object $0$. In Boolean algebra, the `implication'
    $a\to b$ is interpreted as $-a\cup b$, but in Heyting algebra,
    it is the exponential rule $b^a$ (written as $a\Imply b$) satisfying
    the adjunction rule $a\cap b\le c$ iff $a\le b\Imply c$. The exponential
    means the fail of LEM(law of excluded middle) unlike Boolean algebra.
    The negation is alias to $-a=a\Imply 0$. 
\end{remark}
\begin{remark}
    A Heyting algebra is obviously a CCC(Cartesian Closed Category).
    In such a category, we have the following canonical isomorhpism.
    $$
        (A+B)\times C\cong(A\times C)+(B\times C)
    $$
    Thus a Heyting algebra is actually a distributive lattice.
\end{remark}

Of course, there should be a Heyting algebra that is not a Boolean
algebra. Think about the poset $\opensets(\real^2)$ (under set inclusion 
$\subseteq$) of all open sets of the plane $\real^2$. Since 
$A\Imply B=\text{Int}(-A\cup B)$, it is not a Boolean algebra. 

\newcommand{\Heyting}{\mathcal{H}}
Now we can attatch to IPC a semantics of Heyting algebra.
Given Heyting algebra $\Heyting$ and $v: V\to \Heyting$, we
extend it to $\formula$ as follows.
\begin{itemize}
    \item $\bar{v}(A)=v(A)$ for all $A\in V$.
    \item $\bar{v}(\bot)=0$.
    \item $\bar{v}(a\vee b)=\bar{v}(a)\cup \bar{v}(b)$.
    \item $\bar{v}(a\wedge b)=\bar{v}(a)\cap \bar{v}(b)$.
    \item $\bar{v}(a\to b)=\bar{v}(a)\Imply \bar{v}(b)$.
\end{itemize}
(The bar may be omitted.)

Let me introduce the following notations (the relation $\models$).
\begin{itemize}
    \item $\Heyting,v\models \varphi$ iff $v(\varphi) = 1$.
    \item $\Heyting\models\varphi$ iff $\Heyting,v\models\varphi$ for all $v$.
    \item $\Heyting,v\models \Gamma$ iff $\Heyting,v\models \varphi$ for all $\varphi\in\Gamma$.
    \item $\Heyting\models\Gamma$ iff $\Heyting,v\models\Gamma$ for all $v$.
    \item $\models\varphi$ iff $\Heyting,v\models\varphi$ for all $\Heyting, v$.
    \item $\Gamma\models\varphi$ iff $\Heyting,v\models\Gamma$ implies $\Heyting,v\models\varphi$ for all $\Heyting, v$
\end{itemize}

\begin{theorem} (Soundness and Completeness)
    In IPC, $\Gamma\proves\varphi$ iff $\Gamma\models\varphi$.
\end{theorem}
\newcommand{\Lindenbaum}{\mathcal{L}}
\begin{proof}
    The soundness is obvious. For the completeness, just like the
    completeness for CPC or first-order logic, we define
    the {\it Lindenbaum algebra} $\Lindenbaum_\Gamma$.
    Define $\sim$ to be the equivalence relation:
    $$
        \theta\sim\psi\text{ iff }\Gamma\proves\theta\to\psi
        \text{ and }\Gamma\proves{\psi\to\theta}.
    $$
    Define a poset $\Lindenbaum_\Gamma=\formula/_\sim=\{[\theta]_\sim
    \mid \theta\in\formula\}$ with order $[\theta]_\sim$ iff $\Gamma
    \proves\theta\to\psi$. Other facilities of a Heyting algebra are
    clearly:
    \begin{itemize}
        \item $[a]_\sim\cup[b]_\sim=[a\vee b]_\sim$
        \item $[a]_\sim\cap[b]_\sim=[a\wedge b]_\sim$
        \item $[a]_\sim\Imply[b]_\sim=[a\to b]_\sim$
        \item $-[a]_\sim=[\neg a]_\sim$
        \item $0=[\bot]_\sim$
        \item $1=[\bot\to\bot]_\sim$
    \end{itemize}
    We define $v(\psi)=[\psi]_\sim$. Clearly for all 
    $\psi\in\Gamma$, $\Gamma\proves\psi$. Thus $\Lindenbaum_\Gamma,
    v\models \Gamma$. Since $\Gamma\models\varphi$, we know 
    $v(\varphi)=1=[\bot\to\bot]_\sim$, i.e., $\Gamma\proves (\bot\to\bot)
    \to \varphi$. This forces $\Gamma\proves\varphi$.
\end{proof}
For example {\it Peirce's law} $((p\to q)\to p)\to p$ is not 
intuitionistically valid. (Consider the algebra of $\opensets(\real^2)$)

Actually we can prove the following to write a efficient computer
programm to check whether a formula is valid or not (decidability).
\begin{theorem}
    To verify whether a formula of length $n$ is valid, it suffices
    to verify it in all Heyting algebras of cardinality less than or
    equal to $2^{2^n}$.
\end{theorem}
\begin{proof}
    Please refer to \cite{Curry-Howard}.
\end{proof}

\section{Kripke Model}
There's also another way to give semantics for IPC, i.e., the
Kripke model. 
\newcommand{\Kripke}{\mathcal{K}}
\begin{definition}
    A Kripke model $\Kripke$ is a poset $K$($\le$) of possible worlds 
    together with a relation $\Vdash$ between each world and a 
    propositional variable, i.e., $\Kripke=\angletuple{K,\le,\Vdash}$ 
    such that
    \begin{itemize}
        \item for $w\in K$ and variable $A$, if $w\Vdash A$ then for 
        all $w'\ge w$, $w'\Vdash A$;
        \item $w\Vdash \varphi\vee\psi$ iff $c\Vdash\varphi$ or $c\Vdash\psi$;
        \item $w\Vdash \varphi\wedge\psi$ iff $c\Vdash\varphi$ and $c\Vdash\psi$;
        \item $w\Vdash \bot$ never happens;
        \item $w\Vdash\varphi\to\psi$ iff $w'\Vdash\psi$ for all $w'\ge w$ such that $w'\Vdash\phi$.
    \end{itemize}
\end{definition}
We write $\Kripke\Vdash \varphi$ for ``for all $w\in\Kripke$, $w\Vdash\varphi$''.

\begin{lemma}
    It can be shown that for a Kripke model $\Kripke$,
    \begin{itemize}
        \item $w\Vdash\neg\varphi$ iff $w'\not\Vdash\varphi$ for all $w'\le w$;
        \item If $w\le w'$ and $w\Vdash\varphi$, then $w'\Vdash\varphi$.
    \end{itemize}
    \admitted
\end{lemma}

\begin{definition}
    A {\it filter} in Heyting algebra $\Heyting$ is a nonempty subset $F$
    closed under $\cap$ and $\le$, i.e.,
    \begin{itemize}
        \item $a,b\in F$ implies $a\cap b\in F$;
        \item $a\in F$ and $a\le b$ implies $b\in F$. 
    \end{itemize}
\end{definition}

Just like the ideals in a ring, we say a proper filter is {\it prime}
provided $a\cup b\in F$ implies $a\in F$ or $b\in F$. Analogous to
the existence of maximal ideal, we can prove that for each proper filter
$F\subseteq\Heyting$ with $a\not\in F$, there exists a prime filter
$G\subseteq\Heyting$ such that $a\not\in G$. 

The set of all prime filters of a Heyting algebra can be used to define a
Kripke model:
\begin{lemma}
    Given $\Heyting$ and assignment $v$, we can define a Kripke
    model $\Kripke$ such that $\Heyting,v\models \varphi$ iff 
    $\Kripke\models\varphi$.
\end{lemma}
\begin{proof}
    Let $\Kripke=\angletuple{<,\le,\Vdash}$ where $K$ is the set of
    all prime filters in $\Heyting$. $\le$ is set inclusion. And
    we define $F\Vdash A$ iff $v(A)\in F$ for all filters $F$ and
    variable $A$. 
\end{proof}

Thus we have the completeness and soundness for Kripke semantics.
\begin{theorem}
    $\Gamma\proves\varphi$ iff for all Kripke models $\Kripke$,
    $\Kripke\Vdash\Gamma$ implies $\Kripke\Vdash\varphi$.
    \admitted
\end{theorem}

An interesting application is that if $\proves \varphi\vee\psi$, then
either $\proves \varphi$ or $\proves\psi$. 

\null

The $\Vdash$ relation for possible worlds can be viewed as a
category of variable, i.e. a map $\formula\to \two^K=
\Hom_\Pos(K,\two)$. We can generalize it as a
cartesian closed category of the form $\Sets^I$ for a poset $I$. 
For example, a Kripke model for group theory is an $I$-indexed
group. if $I=\opensets(X)\op$ for some topological space $X$,
then this will be a presheaf of groups. Unfortunately, this
part is also vaguely described in \cite{cat-awodey}. I think
I will go ahead with the book \cite{Introduction-to-higher-order-categorical-logic}
for this topic. 

\section{Glivenko's Theorem}
As promised before, I will prove the double negation translation.
The theorem is constructive and IPC is constructive. Thus the
theorem gives a constructive inspection of a traditional proof.
The deduction system I present in \autoref{Mendelssohn-system} 
(Mendelssohn's variation of Hilbert system) uses
`{\it proof by contradiction}' as a third logical axiom. The first
two axioms guarantee the deduction theorem (IPC tautologies).
Actually in IPC (or in CPC without the third axiom), 
the axiom `proof by contradiction' is equivalent to the law of
excluded middle (or double negation law $\neg\neg p\to p$). The
theorem turns every proof in CPC into constructive ones except
that the double negation law is not constructive. This is the failure
of law of excluded middle as value in Heyting algebra ($p\vee\neg p$
is no longer mapped to $1$ in Heyting algebra). For convenience,
when referring to a {\it classical tautology}, I mean a formula
derivable from the Mendelssohn system, but with the third logical
axiom (proof by contradicton) replaced by Law of Excluded Middle. 

\begin{lemma}
    (Affirmation of the consequent)
    $$
    \proves a\to(b\to a)
    $$
\end{lemma}
\begin{proof}
    $$
    \infer{\proves a\to(b\to a)}{\infer{
        a\proves b\to a
    }{a,b\proves a}}
    $$
\end{proof}

\begin{lemma}
    (Distributivity of Implication)
    $$
    \proves (a\to(b\to c))\to ((a\to b)\to (a\to c))
    $$
\end{lemma}
\begin{proof}
    Let $\Gamma=\{a\to(b\to c), a\to b\}$.
    $$
    \infer{\proves (a\to(b\to c))\to((a\to b)\to (a\to c))}{
        \infer{a\to(b\to c)\proves (a\to b)\to (a\to c)}{
            \infer{\Gamma\proves a\to c}{\infer{\Gamma,a\proves c}{
                \infer{\Gamma,a\proves b}{
                    \Gamma,a\proves a\to b & \Gamma,a\proves a
                }
                &
                \infer{\Gamma,a\proves b\to c}{
                    \Gamma,a\proves a & \Gamma,a\proves a\to(b\to c)
                }
            }}
        }
    }
    $$
\end{proof}

\newcommand{\unit}{\name{unit}}
\begin{lemma}
    \label{unit-ipc}
    (Unit) For formula set $\Gamma$ and formula $\varphi$
    $$
        \infer{\Gamma\proves \neg\neg \varphi}{\Gamma\proves \varphi}.
    $$
    ($\proves \varphi\to\neg\neg\varphi$)
    Or, in $\lambda$-calculus(Curry-Howard Correspondence), this a function
    $$
        \unit_\varphi: \varphi \to (\varphi\to \zero)\to \zero.
    $$
\end{lemma}
\begin{proof}
    $$
        \infer{\Gamma\proves \neg\neg \varphi}{
            \infer{\Gamma,\neg \varphi\proves\bot}{
                \Gamma,\neg \varphi\proves\neg \varphi &
                \Gamma,\neg \varphi\proves \varphi
            }
        }
    $$
    $$\unit_\phi(x:\phi, f:\phi\to\zero) \defn f(x):\zero$$
    $$\text{or }\unit_\phi = \lambda x.\lambda f. f\ x$$
\end{proof}

\newcommand{\bind}{\name{bind}}
\begin{lemma}
    \label{bind-ipc}
    (Bind)
    $$
        \infer{\Gamma\proves\neg\neg b}{
            \Gamma,a\proves\neg\neg b
             & \Gamma\proves\neg\neg a
        }
    $$
    Or $\bind_{a,b}:((a\to \zero)\to\zero)\to(a\to((b\to \zero)\to \zero))\to (b\to \zero)\to \zero$
    ($\neg\neg a\to (a\to \neg\neg b)\to \neg\neg b$).
\end{lemma}
\begin{proof}
    $$
    \infer{\Gamma\proves\neg\neg b}{\infer{\Gamma,\neg b\proves\bot}{
        \Gamma\proves\neg\neg a &
        \infer{\Gamma,\neg b\proves\neg a}{
            \infer{\Gamma,a,\neg b\proves\bot}{
                \Gamma,a,\neg b\proves\neg b
                &
                \Gamma,a,\neg b\proves\neg\neg b
            }
        }
    }}
    $$

    $$
    \bind_{a,b}(x:(a\to\zero)\to \zero, f: a\to(b\to \zero)\to\zero)
    \defn \lambda k. x\ (\lambda z. f\ z\ b).
    $$
    Note that $k: b\to \zero$ and $z: a$. Thus $f\ z:(b\to\zero)\to\zero$
    and $(f\ z)\ k$ (or $f(z,k)$) has type $\zero$. Then the term
    $\lambda z.f\ z\ b: a\to\zero$. Applying $x$ to this, we get $\zero$.
    $k$ is called continuation and this function is used in 
    \autoref{monad-examples} later.
\end{proof}
\begin{theorem}
    \label{glivenko-theorem}
    (Glivenko's Theorem) A formula $\varphi$ is a classical tautology
    iff $\proves\neg\neg\varphi$ in IPC. 
\end{theorem}
\begin{proof}
    One direction is trivial. 
    I've already shown that $\proves \neg\neg(p\vee\neg p)$ 
    \eqref{nnLEM}.
    Suppose the proof sequence towards $\varphi$ is $\varphi_1,\cdots,
    \varphi_n$ and $\varphi_n=\varphi$. By reordering, we can assume
    the proof sequence is actually $p_1\vee\neg p_1,\cdots, 
    p_k\vee\neg p_k$, $\varphi_1,\cdots,\varphi_n$ and $\varphi_i$ are
    derived with intuitionistic tautologies and modus ponens. Thus
    we have $\Gamma\proves \varphi_1$ in IPC, where 
    $\Gamma=\{p_1\vee\neg p_1\cdots,p_k\vee\neg p_k\}$,
    and $\Gamma,\varphi_1\proves\varphi_2$. This implies
    $$
        \infer{\Gamma\proves\varphi_2}{
            \infer{\Gamma\proves\varphi_1\to\varphi_2}{
                \Gamma,\varphi_1\proves\varphi_2
            }
            &
            \Gamma\proves\varphi_1
        }
    $$
    (Note the occurence of $\Gamma,\varphi_1\proves\varphi_2$ should be
    another tree of natural deduction representing the proof of the 
    relation $\proves$.)
    Similarly, we get $\Gamma\proves \varphi$ ($\varphi_n=\varphi$).
    Using \Cref{unit-ipc}, we have $\Gamma\proves\neg\neg\varphi$.
    Then we eliminate $p_i\vee\neg p_i$ with \Cref{bind-ipc}.
    Let $\Gamma_k=\Gamma$, $\Gamma_{i-1}=\Gamma_i\setminus(p_i\vee\neg p_i)$
    $$
        \infer{\proves \neg\neg\varphi}{\infer{\cdots}{
            \infer{\Gamma_{k-1}\proves \neg\neg\varphi}{
                \Gamma_k\proves \neg\neg\varphi
                &
                \Gamma_{k-1}\proves \neg\neg(p_k\vee\neg p_k)
            }
            &
            \Gamma_{k-2}\proves \neg\neg(p_2\vee\neg p_2)
        }}
    $$
\end{proof}

\begin{remark}
    The $\bind$ function is helpful when we try proving something
    with Curry-Howard correspondence, i.e., it helps to generate
    lambda terms efficiently. (Forget about the convention that $A$ 
    represents propositional variables for a while.)
    For example, if we have
    $$ma:\neg\neg A,$$
    $$f: A\to\neg\neg B,$$
    $$mh:\neg\neg(B\to C),$$
    $$g: C\to\neg\neg D,$$
    we can obtain a term of type $\neg\neg D$ easily.
    \eq{
        \label{bind-a-lot}
        & \bind\ ma\ (\lambda a. \\
        & \quad \bind\ (f\ a)\ (\lambda b. \\
        & \quad\quad \bind\ mh\ (\lambda h. \\
        & \quad\quad\quad g\ (h\ b) \\
        & \quad\quad) \\
        & \quad)\\
        & )
    }
    Or may be you prefer this form
    \begin{equation}
        \bind(ma,\lambda a.\bind(f(a),\lambda b.\bind(mh,\lambda h.g(h(b))))).
    \end{equation}
    Try to infer the type of each term in this process. 
    It seems insane to write a proof like this, but actually there's a
    concept called {\it monad} to simplify the process. Let me introduce
    it in the next chapter. 
\end{remark}
