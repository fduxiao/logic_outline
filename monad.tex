\chapter{Monad}
\label{monad}

Let's recall what is an {\it adjunction} in category theory.
\def\mathbi#1{\textbf{\em #1}} 
\newcommand{\Src}{\mathbi{Src}}
\newcommand{\Trg}{\mathbi{Trg}}

\begin{definition}
    An adjunction is a pair of functors $F:\Src\to\Trg$, $G:\Trg\to\Src$ 
    together with two assignments $(\cdot)^\sharp,(\cdot)_\flat$, a pair 
    of mutually inverse bijections (called transposition assignments)
    \eq {
        \Src[A,GS] &\to \Trg[FA, S] \\
        f &\mapsto f^\sharp \\
        \\
        \Trg[FA, S] &\to \Src[A,GS] \\
        g &\mapsto g_\flat
    }
    such that the induced two functors $\Src\op\times\Trg\to\Sets$
    are naturally isomorphic to each other, written as $F\dashv G$. 
    
    Here $\Src[A,B]$ is the Hom-set between objects $A,B$ in category
    $\Src$. A more common notation is $\Hom_\Src(A,B)$. For a functor,
    the object assignment is written as $FA$, while morphism assignment
    is written as $F(f)$. 

    There are many other equivalent definitions of adjunction and 
    notations for category theory. I prefer those in 
    \cite{simmons2011introduction}.
\end{definition}

The naturality can be rewritten as for all

$$
\begin{array}{ccc}
    \Src & \qquad & \Trg \\\\
    B\quad A & \text{objects} & S\quad T \\\\
    B \overset{k} \longrightarrow A & & S \overset{l} \longrightarrow T\\
    & \text{morphisms} & \\
    A \underset{f} \longrightarrow GS & & FA \underset{g} \longrightarrow S\\\\
    (G(l)\circ f\circ k)^\sharp = l\circ f^\sharp\circ F(k) & &
    G(l)\circ g_\flat \circ k = (l\circ g\circ F(k))_\flat
\end{array}
$$

\newcommand{\dual}[2]{
    \begin{multicols}{2}
        \begin{center}
            #1
        \end{center}
        \columnbreak 
        \begin{center}
            #2
        \end{center}
    \end{multicols}
}

\begin{definition}
    Given an adjunction $F\dashv G$, we can form two endofunctors
    $F\circ G$ and $G\circ F$. The transposition assignments give
    the natural transformations between identity functors and them.
    For each
    \dual{$\Src$-object $A$}{$\Trg$-object $S$}
    we define the
    \dual{\sf unit}{\sf co-unit}
    to be the natural transformations
    \dual{$\eta_A=(\id_{FA})_\flat$}{$\epsilon_S=(\id_{GS})^\sharp$}
    \dual{$A\overset{\eta_A}\longrightarrow(G\circ F)A$}
        {$(F\circ G)S\overset{\epsilon_S}\longrightarrow S$}  
\end{definition}

\section{From Adjunction to Monad}
Give $F\dashv G$, we define the endofunctor $T=G\circ F:\Src\to\Trg\to\Src$.
The unit $\eta:\Id\to T$ connects the `codomain' of $T$ with identity.
A good property of this functor is the elimination transformation.
Consider the counit at $FA$ for some object $A$
$$
    \epsilon_{FA}:FGFA\to FA.
$$
Then apply functor G
$$
    G(\epsilon_{FA}):GFGFA\to GFA.
$$
We define ths as a new transformation (Note $GF=T$)
$$
    \mu:T^2\to T
$$
The triple $(T,\eta,\mu)$ is called a {\it monad}\footnote{
Originally the monad is just called a triple as in 
\cite{Introduction-to-higher-order-categorical-logic}, in which it is
also said that Huber proved that an adjunction defines a monad. },
but note that the adjunction also gives you the following commutative diagrams (see \cite{cat-awodey} for a detailed proof): 

$$
\begin{tikzcd}
    T \arrow{r}{T\eta} \arrow{d}{\eta_T} \arrow{rd}{\Id} & T^2
    \arrow{d}{\mu} \\
    T^2 \arrow{r}{\mu} & T
\end{tikzcd}
$$

$$
\begin{tikzcd}
    T^3 \arrow{r}{T\mu} \arrow{d}{\mu_T} & T^2
    \arrow{d}{\mu} \\
    T^2 \arrow{r}{\mu} & T
\end{tikzcd}
$$

Here, $T\eta$ ($T\mu$) means applying the functor $T$ to the morphism 
$\eta_A:A\to TA$ ($\mu_A: T^2A\to TA$), $\eta_T$ ($\mu_T$) means the 
morphism $\eta_{TA}:TA\to T^2A$ ($\mu_{TA}:T^3A\to T^2A$)
induced by the natural transformation. 

\null

If you are familiar with it, the above is the monoidal condition, i.e.
the monad is just a monoidal object in the category of endofunctors!
$$
\begin{tikzcd}
    T\circ T \arrow{r}{\mu} & T & \Id \arrow[swap]{l}{\eta}
\end{tikzcd}
$$

\begin{definition}
    (Monad) A triple $(T,\eta\mu)$ consisting of an endofunctor $T$
    and two natural transformations $\eta:\Id\to T$ and $\mu:T^2\to T$
    is called a monad ($\Id$ is the identity functor) iff 
    $$
        \mu\circ\mu_T=\mu\circ T\mu
    $$
    $$
        \mu\circ\eta_T=\mu\circ T\eta=\Id_T.
    $$
\end{definition}

The monad states an adjunction purely algebraically. The free 
construction, the Hom-sets, exponentials, or some other equivalent of 
adjunction can be replaced by an equational representation. Of course
I have to prove that each monad gives you an adjunction in the sense
of `isomorhpism', but let me first give some examples. 

\bigskip
In a preordered set $(P,\le)$, a monad is a (monotonic) closure 
operation $T:P\to P$ such that for all $A\in P$, $A\le TA$ and 
$TTA\le TA$. \bigskip

\newcommand{\powerset}{\mathcal{P}}
The powerset $\powerset: \Sets\to\Sets$ can be viewed as an endofunctor. 
The unit $\eta_A:A\to\powerset A$ is the singleton set $a\mapsto \{a\}$. 
The join\footnote{I'm not sure whether this has a name. The name `join'
again comes from Haskell.} $\mu:\powerset^2 A\to \powerset A$ is the
union $\mu_A(\mathscr{A})=\cup_{A\in\mathscr{A}}A$. Now let's explore
what adjunction gives this monad. 

\newcommand{\cat}{\mathbf}

\section{From Monad to Adjunction}
\begin{definition}(Resolution Category)
    For a monad $(T,\eta,\mu)$ on a category $\cat{C}$, a
    {\it resolution} $(\cat{D},G,F,\epsilon)$ of it consists
    of a category $\cat{D}$, two functors $F:\cat{C}\to\cat{D}$
    and $G:\cat{D}\to\cat{C}$ such that $G\circ F=T$ and the
    equivalence of counit $\epsilon$ such that $G(\epsilon_{FA})=\mu_A$.
    
    Further, all resolutions form a category whose morphisms
    $\Phi:(\cat{D},G,F,\epsilon)$ $\to(\cat{D'},G',F',\epsilon')$ are 
    functors between the underlying categories such that $\Phi F=F'$,
    $G'\Phi=G$ and $\Phi\epsilon=\epsilon'\Phi$.
\end{definition}

We have an initial object and a terminal object in the resolution
category.

\begin{definition}
    (Algebra and Eilenberg-Moore category) Given a monad $(T,\eta,\mu)$ 
    on $\cat{C}$, a {\it $T$-algebra} is a pair $(A,\alpha)$ consisting
    of an object $A\in\cat{C}$ and a morphism $\alpha: TA\to A$ such that
    $\id_A=\alpha\circ\eta_A$ and $\alpha\circ\mu_A=\alpha\circ T\alpha$.
    $$
    \begin{tikzcd}
        A \arrow{r}{\eta_A} \arrow{rd}{\id} & TA\arrow{d}{\alpha}
        & T^2A \arrow{r}{T\alpha}\arrow{d}{\mu_A} & TA\arrow{d}{\alpha}\\
        & A & TA\arrow{r}{\alpha} & A
    \end{tikzcd}
    $$
    The {\it Eilenberg-Moore} category $\cat{C}^T$ is the category of 
    all $T$ algebras whose morphisms $h: (A,\alpha)\to(B,\beta)$ are 
    the original morphisms $h: A\to B$ in $C$ such that 
    $h\circ\alpha=\beta\circ T(h)$.
    $$
    \begin{tikzcd}
        TA\arrow{d}{\alpha} \arrow{r}{Th} & TB\arrow{d}{\beta} \\
        A \arrow{r}{h} & B
    \end{tikzcd}
    $$
    Define $G:(A,\alpha)\mapsto A$. Then $F\dashv G$ gives you the monad.
\end{definition}\bigskip

Let $(A,\alpha)$ be an algebra in $\Sets$. Note the unit and join
\eq{
    \eta_A:A\to\powerset A \\
    a\mapsto \{a\} \\
    \\
    \mu_A:\powerset^2A\to\powerset A\\
    \mathscr{A}\mapsto\bigcup{\mathscr{A}}
}

Then we have $$\alpha(\eta_A(a))=\alpha(\{a\})=\id_A(a)=a$$ and
$$\alpha(\mu(\mathscr{A}))=\alpha T\alpha(\mathscr{A})=\{\alpha(S)\mid
S\in\mathscr{A}\}.$$ If we define $a\le b$ iff $\alpha(\{a,b\})=b$,
then the two properties ensures each algebra is a complete lattice
$(A,\le)$. The morphisms between them means $\sup$-preserving maps. 

\begin{definition}
    \label{kleisli-category}
    The Kleisli Category $\cat{C}_T$ of a monad $(T,\eta,\mu)$ on 
    $\cat{C}$ is a category whose objects are the objects of $\cat{C}$ 
    and there is a morhpism $f: A\to B$ iff there is a morphism $A\to TB$
    in $\cat{C}$. The composition of $f:A\to B$ and $g:B\to C$ is
    equivalent to finding a $g\ast f:A\to TC$, given $f:A\to TB$ and 
    $g:B\to TC$. Since $T(g): TB\to TTC$ and $\mu_C:TTC\to TC$, we have
    $T(g)f: A\to TTC$ and $g\ast f=\mu_CT(g)f$. Clearly the identity
    is the unit $\eta$. Define $G_T:\cat{C}_T\to \cat{C}$ by
    $$
        G_TA = TA,\quad G_T(f:A\to B)=\mu_BT(f)
    $$
    and $F_T: \cat{C}_T\to\cat{C}_T$ by
    $$
        F_T(A)=A,\quad F_T(f:A\to B)=\eta_Bf.
    $$
    Note that a morphism $f: A\to B$ in $\cat{C}_T$ is equivalent to a
    morphism $f:A\to TB$ in $\cat{C}$, and $G_T(f): TA\to TB$. (Recall
    the $\bind$ function (\cref{bind-ipc}). Given an implication 
    formula $A\to \neg\neg B$ and $\neg\neg A$, we can prove $\neg\neg 
    B$.) It can be verified that $F_T\dashv G_T$ gives you the monad.
\end{definition}

\begin{theorem}
    $\cat{C}^T$ is a terminal object in the category of resolutions,
    while $\cat{C}_T$ is an initial object. In fact, the Kleisli category
    is the category of all free algebras as a full subcategory of 
    Eilenberg-Moore category.
\end{theorem}
\begin{proof}
    Please refer to \cite{Introduction-to-higher-order-categorical-logic}.
\end{proof}

Let's find the Kleisli category of the power set. Given $f: A \to 
\powerset{B}$ and $g: B\to\powerset{C}$, the `composition' $g\ast 
f:A\to\powerset{C}$ is a function $(g\ast f)(a)=\mu_C(\powerset(g)(f(a)))
=\bigcup_{b\in f(a)}g(b)$.

If your morphism is `function-like' (a set-theoretic function or a
functon type), then you may have noticed that the key fact behind a
composition and the functor $G_T$ is a `function' of type
$(A\to TB)\to TA\to TB$. Thus we can define a monad like this. 

\begin{definition}
    A Kleisli triple $(T,\eta,(-)^\ast)$ on a category $\cat{C}$ 
    consists of 
    \begin{itemize}
        \item a correspondence $T$ between objects of $\cat{C}$ (which
        does not need to be an endofunctor), 
        \item an assignment $\eta$ that for
        each object $A$, there's a morphism $\eta_A: A\to TA$
        \item a lift mapping every $f:A\to TB$ to $f^\ast:TA\to TB$
    \end{itemize}
    such that
    \begin{itemize}
        \item $\eta_A^\ast=\id_{TA}$,
        \item for $f:A\to TB$, $f^\ast\circ\eta_A=f$,
        \item for $f:A\to TB$, $g:B\to TC$, $(g^\ast\circ f)
        ^\ast=g^\ast\circ f^\ast$.
    \end{itemize}
    With these properties, we can define $(\eta_B\circ f)^\ast:TA\to TB$
    for some $f:A\to B$, making $T$ into a functor and $\eta$ is a
    natural transformation (a lot to verify). 
    Further, we can define $\mu_A=(\eta_A^\ast)^\ast=(\id_{TA})^\ast: 
    T^2A\to TA$, making it into a monad. 
\end{definition}

In fact my definition of the $\bind$ function is a reordering of the lift
in a Kleisli tuple. This definition gives rise to an interesting 
syntactic application.

\section{Monadic Syntax}

For a category of lambda calculus, if an endofunctor $T$ gives
a monad, or the Kleisli tuple is given directly, we write
\eq{
    &a \leftarrow m \\
    &N
}
for $(\lambda a. N)^\ast(m)$ or $\bind(m,\lambda a.N)$.
Suppose the type of $m$ is $TA$, the type of $N$ is $TB$, then the
term $\lambda a.N$ must have type $A\to TB$. This is equivalent
to forcing $a$ to have a type $A$.

For example the proof \eqref{bind-ipc}
$$
\bind(ma,\lambda a.\bind(f(a),\lambda b.\bind(mh,\lambda h.g(h(b)))))
$$
can be reorganized as
\eq{
    & a \leftarrow ma \\
    & b \leftarrow f(a) \\
    & h \leftarrow mh \\
    & g(h(b)).
}

A construction of Glivenko's theorem can also be given as
\eq{
    & lem_{p_1} \leftarrow ({\text some proof}_1: \neg\neg(p_1\vee\neg p_1)) \\
    & lem_{p_2} \leftarrow ({\text some proof}_2: \neg\neg(p_2\vee\neg p_2)) \\
    & \cdots \\
    & \eta({\text some\ result}).
}

The rule for this monadic syntax can be summarized as
\begin{itemize}
    \item each line is a $\leftarrow$ where the term on the
    right must have type $TA$, and the variable (used in the lambda
    abstraction) must have type $A$. 
    \item the last line must have type $TB$ and the result of the 
    the whole expression is of type $TB$. (Note you may want to use
    the unit $\eta$ to make it reasonable).
\end{itemize}

If you understand the functor $T$ as `{\it encapsulating some 
information},' then the $\bind$ function helps you to {\it retrieve}
it and the infomation is {\bf bound} to the variable before the 
$\leftarrow$. The notation (or monad) is much more widely applied.
You can find some other interesting examples in \autoref{monad-examples} about it. 
