\chapter{Classical Logic}
\label{classical-logic}
Let's first discuss an interesting problem (the L\"{o}wenheim–Skolem theorem).
Suppose $f:U\times U\to U$ and $g:U\to U$ are two functions. A set
$K\subseteq U$ is called {\it closed} if $f(K,K)\subseteq K$ and 
$g(K)\subseteq K$. Given an $E\subseteq U$, the closure $\bar{E}$ is
the smallest closed set containing $E$, i.e. $E\subseteq \bar{E}$ and
$\forall K\subseteq U$, if $K$ is closed and $E\subseteq K$, then $\bar{E}
\subseteq K$. Define 
$$
    E^\ast = \bigcap\{F\subseteq U\mid E\subseteq F\text{ and $F$ is closed}\}.
$$

Define $C_0=E$, and when $C_n$ is defined, we define $C_{n+1}$ as
$$
    C_{n+1} = C_n\cup f(C_n, C_n)\cup g(C_n)
$$
and let
$$
    E_\ast = \bigcup_{n\in\nat}C_n
$$
Clearly $\bar{E}=E^\ast=E_\ast$.

This example is how recursion works in set theory, or the explanation of
{\it free} in set theory. Actually it means some initial algebra 
\cite{homotopy-type-theory} and is related to monad 
\cite{reddy1995monads,jacobs1997tutorial}.

\section{Well Formed Formula(WFF)}

Let $V$ be a set of {\it propositional variables}. The members of this
set are denoted as capital letters $A,B,C,...$. For a {\it formula} as
the concept defined below, I use lowercase Greek letters $\varphi,\psi,\theta$
or lowercase English letter $a,b,c$. For example $a\to b$ can be 
$(A\to C)\to(\neg B)$ or $A\to (B\to C)$, but $A\to B$ is just $A\to B$.

A {\bf well formed formula}
is a member of the recursively defined set

$$
    \formula ::= 
        V 
        \mid \formula \to \formula
        \mid \neg \formula
$$
In some book, only $\formula \vee \formula$ and $\neg \formula$ are defined
and $A \to B$ is alias to $\neg A \vee B$. This is merely a style difference.
More logical conjunctions means more syntaxes in the deduction system. 

\section{Semantics of WFF}
Let $\two=\{\zero,\one\}$. An assignment is a function $v: V\to \two$, and
its extension $\bar{v}:\formula \to \two$ is a function such that
$\bar{v}(A)=v(A)$, $\bar{v}(\neg a)=1 - \bar{v}(a)$ and 
$\bar{v}(a\to b)=\max\{1-\bar{v}(a),\bar{v}(b)\}$.
Thus each formula or logical conjunction 
can be viewed as a function $\two^n\to\two$. You can
also define other logical conjunctions such as $\bar{v}(a\wedge b)=
\min\{\bar{v}(a),\bar{v}(b)\}$, $\bar{v}(a\vee b)=
\max\{\bar{v}(a),\bar{v}(b)\}$. You may notice that the implication 
$a\to b$ is identical to not $a$ or $b$. If you interpret $\one$ as
true and $\zero$ as false, this means an implication is true iff
the antecedent is wrong or the consequent is correct. It's very
weired for beginners, but think about this example  
``The Chinese football team has never lost a point in the World Cup
in the past decade''. (This year is 2020.) It's true because they've
never entered world cup. The equivalence of $a\to b$ and $\neg a\vee b$
is equivalent to the {\bf Law of Excluded Middle} (LEM) $p\vee\neg p$.
This makes a classical deduction system complete, but is not true
for intuitionistic logic. 

Given a set of logical conjunctions, if any
function $\two^n\to\two$ can be represented by a formula with only
logical conjunctions in that set, then the set is called {\it
functionally complete}. Clearly $\{\neg,\to\}$ is functionally
complete.

\section{Deduction system}
\label{Mendelssohn-system}
Hilbert once developed a strict syntax about what is a proof. I
give the variation of Mendelssohn. Let $\Gamma$ be a set of wffs.
A proof from $\Gamma$ to $\varphi$ is a finite sequence $\varphi_n$ of
wffs with the last one $\varphi$ such that each $\varphi_n$ comes from
one of the following situations.

\begin{enumerate}
    \item $\varphi_n\in\Gamma$
    \item $\varphi_n$ has one of the following forms 
        \begin{enumerate}
            \item $a\to (b\to a)$ (Affirmation of the consequent)
            \item $(a\to(b\to c))\to((a\to b)\to(a\to c))$ (Distribution of implication)
            \item $(\neg a\to\neg b)\to((\neg a\to b)\to a)$ (Proof by contradiction)
        \end{enumerate}
    \item There exists $i,j<n$ and $\varphi_j=\varphi_i\to\varphi_n$ (MP, Modus Ponens)
\end{enumerate}

If a proof from $\Gamma$ to $\varphi$ exists, we write $\Gamma\proves\varphi$.
If $\Gamma=\emptyset$, we write $\proves\varphi$. 
For example, the following is a proof sequence towards $p\to p$.
\begin{enumerate}
    \item $p\to((p\to p)\to p)$ (2.a)
    \item $(p\to((p\to p)\to p))\to((p\to(p\to p))\to(p\to p))$ (2.b)
    \item $(p\to(p\to p))\to(p\to p)$ (1,2 MP)
    \item $p\to(p\to p)$ (2.a)
    \item $p\to p$ (3,4 MP)
\end{enumerate}

\begin{theorem}
    \label{deduction-theorem} (Deduction Theorem)
    $\Gamma,a\proves b$ iff $\Gamma\proves a\to b$.
\end{theorem}
\begin{proof}
    Clearly if $\Gamma\proves a\to b$, then $\Gamma,a\proves a\to b$ and
    also we have $\Gamma,a\proves a$. Using MP, we have $\Gamma,a\proves b$.

    Suppose conversely $\Gamma,a\proves b$. We prove $\Gamma\proves a\to b$
    by induction on the proof sequence. Suppose $\varphi_1,\cdots,\varphi_n$
    is a proof of $b$ from $\Gamma\cup\{a\}$, i.e. $\varphi_n=b$.
    When $n=1$, then $b$ is obtained by
    \begin{case}
        $b\in\Gamma\cup\{a\}$. \\
        If $b\in\Gamma$, then $\Gamma\proves b$ and 
        $\Gamma\proves b\to a$. This implies $\Gamma\proves a\to b$ \\
        If $b=a$, then certainly $\Gamma\proves a\to a$
    \end{case}
    
    \begin{case}
        $b$ is a logical axiom. Thus $\Gamma\proves b$ and 
        $\Gamma\proves b\to(a\to b)$. Using MP we have $\Gamma\proves a\to b$
    \end{case}

    Then we suppose this is true for all proof sequences with length
    less than $n$. The last formula comes from either the axiom set $\Gamma$
    or the logic axioms. For the former, It has already been shown, so we
    suppose $b$ is obtained by MP, i.e., we can find $i,j<n$ such that 
    $\varphi_i=c\to b$ and $\varphi_j=c$. By definition, this is 
    $\Gamma,a\proves c$ and $\Gamma,a\proves c\to b$. Applying the
    inductive hypothesis, we have $\Gamma\proves a\to c$, 
    $\Gamma\proves a\to(c\to b)$. Note we also has a logical axiom
    $\Gamma\proves (a\to(c\to b))\to((a\to c)\to (a\to b))$.
    Applying MP twice, we derive $\Gamma\proves a\to b$.
\end{proof}

\section{Soundness and Completeness}
An assignment $v:V\to \two$ satisfies $\Gamma$ if for any $\varphi\in\Gamma$,
$\bar{v}(\varphi)=\one$. For a wff $\varphi$, if for any assignment 
$v:V\to\two$ satisfiying $\Gamma$, $\bar{v}(\varphi)=\one$, then $\varphi$ is
called the semantic consequent of $\Gamma$, written as $\Gamma\models\varphi$.

\begin{theorem}
    The deduction system is sound, i.e., $\Gamma\proves\varphi$ implies
    $\Gamma\models\varphi$. 
\end{theorem}

\begin{theorem}
    The deduction system is complete, i.e., $\Gamma\models\varphi$ implies
    $\Gamma\proves\varphi$. 
\end{theorem}

These two relations are the core concern of a logic system. The soundness
is very easy to verify, but the proof of completeness is often tricky.
The soundness ensures each proof is meaningful and the completeness means
that you can always find a `proof' as defined above. This can be extended
to a ``first order logic''. For example, we can write a pure syntactic
proof of ``groups of order prime sqaure are abelian'', but a semantic
proof is more convenient. The semantic part, as a first order part in
the meta language of meta language, is complete. Thus the completeness
ensures us to find a proof (where a group is defined as a set). 

Here's a last exercise, you may want to know why we need these 3
logical axioms. Though they ensure the completeness, are they all
necessary? For example, can one derive 
$(\neg a\to\neg b)\to((\neg a\to b)\to a)$ from the other two axioms? 
The answer is negative. {\it Hint}: think about the triple value logic
below. 

\begin{center}
    \begin{tabular}{c|c}
        $\neg$ &  $a$ \\
        \hline
        0 & 0 \\
        0 & 1 \\
        0 & 2 \\
    \end{tabular}

    \begin{tabular}{c | c | c}
         $a$ &   $\to$   & $b$ \\
         \hline
          0  &   0   &  0  \\
          0  &   0   &  1  \\
          0  &   2   &  2  \\
          1  &   0   &  0  \\
          1  &   0   &  1  \\
          1  &   2   &  2  \\
          2  &   0   &  0  \\
          2  &   0   &  1  \\
          2  &   0   &  2  \\
    \end{tabular}
\end{center}

