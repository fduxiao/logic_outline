\chapter{Homotopy Type Theory}
\label{HoTT}

In this chapter I mainly describe the `$=$' type and how it raises
a homotopy theory about types. First let's recall some homotopy theory
in algebraic topology.

\section{Homotopies}

This section is about algebraic topology. All `maps' are assumed to be
continuous unless mentioned otherwise. 

\begin{definition}
    \label{path}
    A {\it path} in space $X$ between two points is a map $p: I\to X$.
    If $p_1,p_2$ are two paths such that $p_1(1)=p_2(0)$, then we
    define their concatenation $p_1\centerdot p_2: I\to X$
    $$
        (p_1\centerdot p_2)(t) = \begin{cases}
            p_1(2\cdot t) & t\in[0,\frac{1}{2}] \\
            p_2(2\cdot(t-\frac{1}{2})), t\in[\frac{1}{2}, 1]
        \end{cases}
    $$
\end{definition}

\begin{definition}
    \label{homotopy}
    Given topological spaces $X,Y$, and $f,g:X\to Y$, a {\it homotopy} between
    $f$ and $g$ is a (continuous) map $H:X\times I\to Y$ such that
    \eq{
        H(x,0) &= f(x) \\
        H(x,1) &= g(x)
    }
    If $A\subseteq X$ is a subset and $f|_A=g|_A$, we can further require
    that $H(x, t)=f(x)=g(x)$ for $x\in A,t\in I$. Such a homotopy is called
    {\it relative homotopy} (homotopy relative to $A$). 
\end{definition}

We can use homotopy relative to the endpoints to define a equivalent
relation on paths. i.e. $[p_1]=[p_2]$ iff there exists a homotopy 
$H: I\times I\to X$ such that $H(0, t)=p_1(0)=p_2(0)$,$
H(1,t)=p_1(1)=p_2(1)$ and $H(s,0)=p_1(s)$, $H(t,1)=p_2(s)$.

\begin{definition}
    \label{fundamental-groupoid}
    Let $X$ be a topological space. The {\it fundamental groupoid\footnote{
        a groupoid is a category where each morphism has an inverse.
    }} $\pi X$ of it is
    a category where the objects are points in $X$ and morphisms
    between $x,y\in X$ is the equivalence classes $[p]$ of all paths
    $p$ between $x$ and $y$.
    The composition of $[p_1]: x\to y$ and $[p_2]:y\to z$ is
    $[p_2]\circ[p_1]=[p_1\centerdot p_2]$. (Note the different conventions
    of order for `composition in category' and `concatenation of paths'.)
    The identity at $x$ is the constant map $c_x$.
\end{definition}

This kind of definition ignores so-called `higher dimensional structures' 
or `higher order structures' because the morphisms between homotopies are
ignored. Traditionally one can define a homotopy between homotopies to
inspect those structures.

\begin{definition}
    If $f,g:X\to Y$ are maps and $F,G: X\times I\to Y$ are homotopies 
    between them, a homotopy between $F$ and $G$ is a map 
    $H: X\times I\times I\to Y$ such that
    $$
    \begin{array}{lcl}
        H(x,s,0) = F(x, s) & & H(x, s, 1) = G(x, s) \\
        H(x, 0, t) = f(x) & & H(x,1,t) = g(x).
    \end{array}
    $$
    This is to say $H$ is a homotopy between $F$ and $G$ and for each
    $t$, $H$ is still a homotopy between $f$ and $g$.
\end{definition}

If you believe a point in $X$ is a map $\one\to X$ and a path between
different points is a homotopy $\one\times I\to X$, then a homotopy
between two paths $p_1,p_2$ between $x,y$ relative to the endpoints
can be understood as a homotopy between homotopies.  
Now We can inspect the higher order homotopy structures by {\it track
groupoid}.

\begin{definition}
    For topological spaces $X$ and $Y$, the {\it track groupoid} $\pi Y^X$
    is a category with objects all continuous maps $X\to Y$ and morphisms
    the equivalence classes of all $homotopies$ between them. 
\end{definition}

The notation $Y^X$ is meaningful because we can equip the set $Y^X=\{
f: X\to Y\mid f\text{ continuous}\}$ with a {\it compact open}\footnote{
More generally, one can give the {\it test open} topology 
\cite{topology-and-groupoids}.} topology satisfying the exponential rule
$Hom(A\times X, Y)=Hom(A, Y^X)$ which is generated by the subbase
$\{V(K, U)\mid K\subseteq X\text{ compact},U\subseteq Y\text{ open}\}$,
where $V(K, U) = \{f: X\to Y\text{ continuous}\mid f(K)\subseteq U\}$.
Without surprise, The fundamental groupoid of the space $Y^X$ is the
track groupoid $\pi Y^X$. 

By this manner, we can definer higher order structures, namely the higher
order homotopy groupoids $\pi_n$. For example $\pi_0$ is the discrete
\footnote{A discrete category is a category where all morphims are
identities.} groupoid of all path components. But we can also define a
weak $\infty$-groupoid\footnote{I have been trying hard to understand
this concept, but it turns out that it requires lots of category theory
including (co)monad, (co)algebra and (co)induction 
\cite{types-are-weak-groupoids} and I've merely found
some fragmentary materials about this. I'm not sure whether
it's a good idea or not that I read the theory technically. Despite
this geometry confusion, I understand how a proof works in 
\cite{homotopy-type-theory}.} where we no longer take the equivalence 
classes and that category consists of points as the objects 
($0$-morphisms), paths as $1$-morphisms, homotopies as $2$-morphism, 
homotopies between homotopies as $3$-morphisms and so on, and the
equality required in identity rule or associativity may be replaced by 
homotopy $\simeq$ (higher order morphisms, namely the weakness).
\newcommand{\sphere}{\mathbb{S}}

The fundamental groupoid helps to classify topological spaces
geometrically. The homotopy intuitively means deformation retraction.
For example, let $\sphere^1=\{x\in\complex\mid |x| = 1\}$ with ordinary 
topology. Let $p(t) = e^{2\pi\iu t}: I\to \sphere^1$ be a path from
$1$ to $1$. There's no evidence that $[p]=[c_x]$ (the endpoints are 
fixed), but in the space ${\sphere^1}^I$ (the endpoints are free), 
they are equivalent (homotopic). Just like the algebraically topological
part of weak $\infty$-groupoid, this gives us an {\it intensional}
\footnote{In contrast to extensional type theory, another type theory
ignoring different possibilities of the equality.} interpretation of types.

\section{Identity Types}

As in the previous chapter, for a type $A:\universe$ and $x,y:A$,
we define a new type $x =_A y$. Without ambiguity, we can omit the $A$,
or write it as $\Id_A(x, y)$. The name `Identity' may cause confusion with
the identity function $\id_A$, so in Coq\cite{coq}, it is called {\it 
equality}. Now I follow the convention in \cite{homotopy-type-theory} still
calling it {\it identity type}.

\begin{definition}
    (Identity Type) The formation rule is 
    $\Id: \Pi_{A:\universe}A\to A\to\universe$ or $\Id_A: A\to A\to \universe$ 
    (note that $\to$ and $\times: \universe\to\universe$ are also 
    `formation rules') with an only constructor 
    $\refl:\Pi_{x:A}x=_A x$ or $\refl_x: x=_A x$ (one of the $x$ may
    be replaced by another judgementally equal term e.g. $0+0\equiv 0$).
\end{definition}

As usual, the induction principle is
$$
    \ind_{=_A}:\prod_{(C:\prod_{(x,y:A)}(x=_A y)\to\universe)}
    (\prod_{x:A}C(x,x,\refl_x))\to\prod_{x,y:A}\prod_{p:x=_A y}C(x,y,p)
$$
with computation rule
$$
    \ind_{=_A}(C,c,x,x,refl_x)\defn c(x).
$$
This rule, the {\it path induction}, is traditionally called $J$ and we 
can derive the {\it leibniz rule} (\autoref{leibniz}) from it. For 
convenience, we have another equivalent variation of it. (This variation 
is called {\it based path induction}.)
$$
    \ind'_{=_A}:\prod_{a:A}\prod_{(C:\Pi_{x:A}(a=_A x)\to \universe)}
    C(a, \refl_a) \to \prod_{x:A}\prod_{p:a=_A x}C(x, p)
$$

$$
    \ind'_{=_A}(a, C, c, a,\refl_a) \defn c.
$$
But be careful that we define the constructor for the space 
$\sum_{x,y:A}x=_A y$ (or $\sum_{y:A}(a=y)$ as for the based induction).
We may prove that $(x,y,p)=_{\sum_{(x,y:A)}(x=y)}(x,x,\refl_x)$, but
this does not mean that $\prod_{p:x=_A x}p=_{x=_Ax}\refl_x$ (because
we cannot even apply the induction rule), or more generally the {\it
Uniquess of Identity Proofs} (UIP).
\newcommand{\uip}{\name{uip}}
$$
\uip: \forall (A: \universe), (x,y:A), (p,q:x=_A y),p=_{x=_A y}q
$$
(It's quite silly to write a lot of $\Pi$s, especially when $\Pi$ exists
in the subscription, so I write $\forall$ instead.)
Note we that cannot define this function by pattern match 
$\uip(A,x,x,\refl,\refl\defn\refl$\footnote{In programming language
idris, this is acceptable.} because we potentially used an
equivalent $K$ axiom \cite{pattern-matching-without-K}
$$
    K: \forall (A:\universe), (a:A), (P: a=a\to\universe),
    P(\refl)\to\forall (loop: a = a), P(loop).
$$

It seems that there's a general way to give `sound' induction rules.
In fact, there's a $W$-type described in chapter 5 of \cite{homotopy-type-theory},
but I haven't reached that chapter. I know another similar application.
Recall the $\Ycomb$ combinator in \autoref{type-theory}, i.e. for any 
(untyped) $\lambda$-term $F$, $F\ (\Ycomb\ F)=_\beta \Ycomb\ F$.
Let $H=\lambda f.\lambda x.M$ and $F = \Ycomb\ H$. Then
\eq{
    F\ y & = (\Ycomb\ H)\ y \\
         & =_\beta H\ (Y\ H)\ y \\
         & = (\lambda f.\lambda x.M)\ (Y\ H)\ y \\
         & = (\lambda f.\lambda x.M)\ F\ y \\
         & = M[f:=F,x:=y].
}
$M[f:=F,x:=y]$ shows that the occurance of $f$ in $M$ will be the `defined'
(or not) $F$, i.e. this is a way to write recursive functions.
For example, if $M$ is something like $\mathbf{IF}\ C\ \mathbf{THEN}\ x
\ \mathbf{ELSE}\ f\ (\succn\ x)$, then $F\ c_n=_\beta M[f:=F,x:=c_n]=
\mathbf{IF}\ C\ \mathbf{THEN}\ c_n\ \mathbf{ELSE}\ F\ (\succn\ c_n)$.

\section{Homotopical Interpretation}
So how shall we understand the failure of $\uip$ rule? In 1996,
Martin Hofmann and Thomas Streicher proposed a homotopical model of
it \cite{the-groupoid-Interpretation-of-type-theory}. Instead of
defining a topological space and calculating its fundamental groupoid,
we define a homotopy theory synthetically, like the Euclidean geometry
versus analytic geometry based on $\mathbb{R}$. Given $p,q:a=_A a$,
there's no evidence that $p=_{a=_Aa}q$, which looks like the path
$t\mapsto e^{2\pi\iu t}$ is not homotopic to $c_1$. Each type $A$
is understood as a space with $p:a=_A b$ a path between two points in
this spaces. The type $a=_A b$ is thus the space of all such paths
from $a$ to $b$, so we can think of $H:p=_{a=_Ab}q$ as a homotopy in
higher order category. Just like the compact open topology, the space
$a=_Aa$ is different from the type $\sum_{y:A}a =_A y$, the space of
all paths from $a$. Thus we cannot assert $p=\refl_a$ and the judgemental
equality is different from this propositional equality. In this pattern,
the structure of the $\infty$-groupoid is given by defining the morphisms,
not from the topology and continuity (defining the hom-sets directly).

To call $p:a=_Ab$ a path, we have to first prove it has an `inverse' and
we can concatenate paths.



It seems that the book of Homotopy Type Theory \cite{homotopy-type-theory} 
wants to interpret the theory syntactically, because it does not require a 
profound comprehension about what is groupoid. Rather, it implies that the
homotopy type theory is a better language for that purpose. For example
when define a
