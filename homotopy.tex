\chapter{Homotopy Type Theory}
\label{HoTT}

In this chapter I mainly describe the `$=$' type and how it raises
a homotopy theory about types. First, let's recall some homotopy theory
in algebraic topology.

\section{Homotopies}

This section is about algebraic topology. All `maps' are assumed to be
continuous unless mentioned otherwise. 

\newcommand{\sqcdot}{\mathbin{\vcenter{\hbox{\rule{.3ex}{.3ex}}}}}
\newcommand{\concat}{\sqcdot}

\begin{definition}
    \label{path}
    A {\it path} in space $X$ between two points is a map $p: I\to X$.
    If $p_1,p_2$ are two paths such that $p_1(1)=p_2(0)$, then we
    define their concatenation $p_1\concat p_2: I\to X$
    $$
        (p_1\concat p_2)(t) = \begin{cases}
            p_1(2\cdot t) & t\in[0,\frac{1}{2}] \\
            p_2(2\cdot(t-\frac{1}{2})), t\in[\frac{1}{2}, 1]
        \end{cases}
    $$
\end{definition}

\begin{definition}
    \label{homotopy}
    Given topological spaces $X,Y$, and $f,g:X\to Y$, a {\it homotopy} between
    $f$ and $g$ is a (continuous) map $H:X\times I\to Y$ such that
    \eq{
        H(x,0) &= f(x) \\
        H(x,1) &= g(x)
    }
    If $A\subseteq X$ is a subset and $f|_A=g|_A$, we can further require
    that $H(x, t)=f(x)=g(x)$ for $x\in A,t\in I$. Such a homotopy is called
    {\it relative homotopy} (homotopy relative to $A$). 

    It can be shown that (relative) homotopy is an equivalence relation.
    We also write $f\simeq g$ if there exists a homotopy between $f$ and
    $g$. 
\end{definition}

We can use homotopy relative to the endpoints to define an equivalent
relation on paths. i.e. $[p_1]=[p_2]$ iff there exists a homotopy 
$H: I\times I\to X$ such that $H(0, t)=p_1(0)=p_2(0)$,$
H(1,t)=p_1(1)=p_2(1)$ and $H(s,0)=p_1(s)$, $H(t,1)=p_2(s)$.

\begin{definition}
    \label{fundamental-groupoid}
    Let $X$ be a topological space. The {\it fundamental groupoid\footnote{
        A groupoid is a category where each morphism has an inverse.
    }} $\pi X$ of it is
    a category where the objects are points in $X$ and morphisms
    between $x,y\in X$ is the equivalence classes $[p]$ of all paths
    $p$ between $x$ and $y$.
    The composition of $[p_1]: x\to y$ and $[p_2]:y\to z$ is
    $[p_2]\circ[p_1]=[p_1\concat p_2]$. (Note the different conventions
    of order for `composition in category' and `concatenation of paths'.)
    The identity at $x$ is the constant map $c_x$.
\end{definition}

This kind of definition ignores so-called `higher dimensional structures' 
or `higher order structures' because the morphisms between homotopies are
ignored. Traditionally one can define a homotopy between homotopies to
inspect those structures.

\begin{definition}
    If $f,g:X\to Y$ are maps and $F,G: X\times I\to Y$ are homotopies 
    between them, a homotopy between $F$ and $G$ is a map 
    $H: X\times I\times I\to Y$ such that
    $$
    \begin{array}{lcl}
        H(x,s,0) = F(x, s) & & H(x, s, 1) = G(x, s) \\
        H(x, 0, t) = f(x) & & H(x,1,t) = g(x).
    \end{array}
    $$
    This is to say $H$ is a homotopy between $F$ and $G$ and for each
    $t$, $H$ is still a homotopy between $f$ and $g$.
\end{definition}

If you believe a point in $X$ is a map $\one\to X$ and a path between
different points is a homotopy $\one\times I\to X$, then a homotopy
between two paths $p_1,p_2$ between $x,y$ relative to the endpoints
can be understood as a homotopy between homotopies.  
Now We can inspect the higher order homotopy structures by {\it track
groupoid}.

\begin{definition}
    For topological spaces $X$ and $Y$, the {\it track groupoid} $\pi Y^X$
    is a category with objects all continuous maps $X\to Y$ and morphisms
    the equivalence classes of all $homotopies$ between them. 
\end{definition}

The notation $Y^X$ is meaningful because we can equip the set $Y^X=\{
f: X\to Y\mid f\text{ continuous}\}$ with a {\it compact open}\footnote{
More generally, one can give the {\it test open} topology 
\cite{topology-and-groupoids}.} topology satisfying the exponential rule
$Hom(A\times X, Y)=Hom(A, Y^X)$, which is generated by the subbase
$\{V(K, U)\mid K\subseteq X\text{ compact},U\subseteq Y\text{ open}\}$,
where $V(K, U) = \{f: X\to Y\text{ continuous}\mid f(K)\subseteq U\}$.
Without surprise, The fundamental groupoid of the space $Y^X$ is the
track groupoid $\pi Y^X$. 

By this manner, the space of `special (continuous) maps between two
spaces' will be meaningful as a subspace. For example, we can talk about
`all paths on the unit circle with one point fixed' as a topological space
and calculate its foundamental groupoid. Besides, the higher order structures
rise naturally as the equivalences between lower morphisms. Instead of
using a strict $=$ relation in category, we define a `weak $\infty$-category'
with $\simeq$ as the equality. For a topological space, we view the
points as `$0$-morphisms', with paths as `$1$-morphisms' between them. It
forms a `weak-category' in the sense that 
$(p_1\concat p_2)\concat p_3\simeq p_1\concat(p_2\concat p_3)$,
where the homotopy $\simeq$ is included in the weak category as a 
`$2$-morphisms' between them. Since each morphism has an inverse, we
can define the fundamental `weak $\infty$-groupoid' of a space. 

\newcommand{\sphere}{\mathbb{S}}
\begin{example}
    Let $\sphere^1=\{x\in\complex\mid |x| = 1\}$ with ordinary 
    topology. Let $p(t) = e^{2\pi\iu t}: I\to \sphere^1$ be a path from
    $1$ to $1$. There's no evidence that $[p]=[c_x]$ (the endpoints are 
    fixed), but in the space ${\sphere^1}^I$ (the endpoints are free), 
    they are equivalent (homotopic). Hence there's no higher order
    structure\footnote{This is not that obvious. The technique to prove
    it is used in \cite{homotopy-type-theory}.} 
    and by the knowledge of fundamental group, we know the fundamental
    weak $\infty$-groupoid of $\sphere$.
\end{example}

Just like the algebraically topological
part of weak $\infty$-groupoid, not every equality between two lambda
terms is proved by reflectivity, which gives us an {\it intensional}
\footnote{In contrast to extensional type theory, another type theory
ignoring different possibilities of the equality.} interpretation of types.

\section{Identity Types}

As in the previous chapter, for a type $A:\universe$ and $x,y:A$,
we define a new type $x =_A y$. Without ambiguity, we can omit the $A$,
or write it as $\Id_A(x, y)$. The name `Identity' may cause confusion with
the identity function $\id_A$, so in Coq\cite{coq}, it is called {\it 
equality}. Now I follow the convention in \cite{homotopy-type-theory} still
calling it {\it identity type}.

\begin{definition}
    \label{identity-type}
    (Identity Type) The formation rule is 
    $\Id: \Pi_{A:\universe}A\to A\to\universe$ or $\Id_A: A\to A\to \universe$ 
    (note that $\to$ and $\times: \universe\to\universe$ are also 
    `formation rules') with an only constructor 
    $\refl:\Pi_{x:A}x=_A x$ or $\refl_x: x=_A x$ (one of the $x$ may
    be replaced by another judgementally equal term e.g. $0+0\equiv 0$).
\end{definition}

As usual, the induction principle is

\begin{equation}
    \label{ind_id}
    \ind_{=_A}:\prod_{(C:\prod_{(x,y:A)}(x=_A y)\to\universe)}
    (\prod_{x:A}C(x,x,\refl_x))\to\prod_{x,y:A}\prod_{p:x=_A y}C(x,y,p)
\end{equation}

with computation rule
$$
    \ind_{=_A}(C,c,x,x,\refl_x)\defn c(x).
$$

You can think of it as `if you want to prove a predication over
the equality $p: x=y$, you only have to prove the situation when 
$y\equiv x$' and $p\equiv \refl$'.

This rule, the {\it path induction}, is traditionally called $J$ and we 
can derive the {\it leibniz rule} (\autoref{leibniz}) from it. For 
convenience, we have another equivalent variation of it. (This variation 
is called {\it based path induction}.)
$$
    \ind'_{=_A}:\prod_{a:A}\prod_{(C:\Pi_{x:A}(a=_A x)\to \universe)}
    C(a, \refl_a) \to \prod_{x:A}\prod_{p:a=_A x}C(x, p)
$$

$$
    \ind'_{=_A}(a, C, c, a,\refl_a) \defn c.
$$

But be careful that we define the constructor for the space 
$\sum_{x,y:A}x=_A y$ (or $\sum_{y:A}(a=y)$ as for the based induction).
We may prove that $$(x,y,p)=_{\sum_{(x,y:A)}(x=y)}(x,x,\refl_x),$$ but
this does not mean that $\prod_{p:x=_A x}p=_{x=_Ax}\refl_x$ (because
we cannot even apply the induction rule, e.g., $n+0=n$ is not a 
reflectivity.), or more generally the {\it
Uniquess of Identity Proofs} (UIP).
\newcommand{\uip}{\name{uip}}
\begin{equation}
    \label{uip}
    \uip: \forall (A: \universe), (x,y:A), (p,q:x=_A y),p=_{x=_A y}q
\end{equation}
(It's quite silly to write a lot of $\Pi$s, especially when $\Pi$ exists
in the subscript, so I write $\forall$ instead.)
Note we that cannot define this function by pattern match 
$\uip(A,x,x,\refl,\refl)\defn\refl$\footnote{In programming language
Idris, this is acceptable.} because we cannot apply the recursor (the
induction rule when $C$ is a constant). In some other theory, an
so-called $K$ axiom is used for pattern matching \cite{pattern-matching-without-K}
\begin{equation}
    \label{K-axiom}
    K: \forall (A:\universe), (a:A), (P: a=a\to\universe),
    P(\refl)\to\forall (l: a = a), P(l).
\end{equation}

It seems that there's a general way to give `sound' induction rules.
In fact, there's a $W$-type described in chapter 5 of \cite{homotopy-type-theory},
but I haven't reached that chapter. I know another similar application.
Recall the $\Ycomb$ combinator in \autoref{type-theory}, i.e. for any 
(untyped) $\lambda$-term $F$, $F\ (\Ycomb\ F)=_\beta \Ycomb\ F$.
Let $H=\lambda f.\lambda x.M$ and $F = \Ycomb\ H$. Then
\eq{
    F\ y & = (\Ycomb\ H)\ y \\
         & =_\beta H\ (Y\ H)\ y \\
         & = (\lambda f.\lambda x.M)\ (Y\ H)\ y \\
         & = (\lambda f.\lambda x.M)\ F\ y \\
         & = M[f:=F,x:=y].
}
$M[f:=F,x:=y]$ shows that the occurence of $f$ in $M$ will be the `defined'
(or not) $F$, i.e. this is a way to write recursive functions.
For example, if $M$ is something like $\mathbf{IF}\ C\ \mathbf{THEN}\ x
\ \mathbf{ELSE}\ f\ (\succn\ x)$, then $F\ c_n=_\beta M[f:=F,x:=c_n]=
\mathbf{IF}\ C\ \mathbf{THEN}\ c_n\ \mathbf{ELSE}\ F\ (\succn\ c_n)$.

\section{Homotopical Interpretation}
So how shall we understand the failure of $\uip$ rule? In 1996,
Martin Hofmann and Thomas Streicher proposed a homotopical model of
it \cite{the-groupoid-Interpretation-of-type-theory}. Instead of
defining a topological space and calculating its fundamental groupoid,
we define a homotopy theory synthetically, like the Euclidean geometry
versus analytic geometry based on $\mathbb{R}$. Given $p,q:a=_A a$,
there's no evidence that $p=_{a=_Aa}q$, which looks like the path
$t\mapsto e^{2\pi\iu t}$ is not homotopic to $c_1$. Each type $A$
is understood as a space with $p:a=_A b$ a path between two points in
this spaces. The type $a=_A b$ is thus the space of all such paths
from $a$ to $b$, so we can think of $H:p=_{a=_Ab}q$ as a homotopy in
higher order category. Just like the compact open topology, the space
$a=_Aa$ is different from the type $\sum_{y:A}a =_A y$, the space of
all paths from $a$. Thus we cannot assert $p=\refl_a$ and the judgemental
equality is different from this propositional equality. In this pattern,
the structure of the $\infty$-groupoid is given by defining the morphisms,
not from the topology and continuity (defining the hom-sets directly).

To call $p:a=_Ab$ a path, we have to first prove it has an `inverse' and
we can concatenate paths.

\begin{lemma}
    For $x,y:A$ and $p: x=y$, we have $p\inv: y=x$, i.e.
    a function $(\cdot)\inv: x=y\to y=x$ such that $\refl_x\inv=
    \refl_x$. $p\inv$ is the inverse of $p$.
\end{lemma}
\begin{proof}
    Let $D(x,y,p)\defn(y=x)$, $D:\prod_(x,y:A)(x=y)\to\universe$ and
    $d\defn(\lambda x.\refl_x):\prod D(x,x,\refl_x)$. Applying the
    induction principle \eqref{ind_id}, we get 
    $p\defn\ind_{=_A}(D,d,x,y,p):(y=x)$. According to the computation
    rule, $\refl_x\inv\defn \ind_{=_A}(D,d,x,x,\refl_x)\equiv d(x)\equiv
    \refl_x$.
\end{proof}
We can also make this proof shorter with {\it pattern matching}.
\begin{proof}
    By induction, it suffices to assume that $x\equiv y$ and $p$ is
    $\refl_x:x=x$. Clearly we can define $p\inv\defn \refl_x:x=x$, (which
    is exactly the computation rule).
\end{proof}

Similarly we can prove the following concatenation of paths.  
\begin{lemma}
    For $x,y,z: A$, $(\concat):(x=y)\to(y=z)\to(x=z)$ such that
    $\refl_x\concat \refl_x\equiv\refl_x$. 
\end{lemma}
\begin{proof}
    You can choose to define a $D(x,y,p)\defn\prod_{z:A}\prod_{q:y=z}(x=z)$
    and so on to invoke the induction principle \eqref{ind_id}, but it is
    a bit too tedious (since this time we need to invoke it twice).
    Instead, let's use pattern matching. By induction on $p$, it suffices 
    to assume $x\equiv y$ and $p\equiv\refl_x$. Now our target is
    $(x=x)\to(x=z)\to(x=z)$. Then let's continue by induction on $q$.
    Thus the target is $(x=x)\to(x=x)\to(x=x)$. We then define the last
    term to be $\refl_x$. 
\end{proof}
You may argue that why not apply id to $(x=x)\to(x=z)\to(x=z)$. The 
difference is that in the proof I presented, the computation rule is 
$\refl_x\concat\refl_x\equiv\refl_x$, while the other proof gives 
$\refl\concat q\equiv q$. Similarly you can define the computation
rule to be $p\concat\refl=p$. That is, as said in 
\cite{homotopy-type-theory}, only a stylistic choice. The other two
should be derived (propositionally) from any of the three computation 
rules and we can have more results.
\begin{lemma}
    Suppose $x,y,z,w:A$ and $p:x=y$, $q:y=z$, $r:z=w$. Then
    \begin{enumerate}
        \item $p=p\concat\refl$, and $p=\refl\concat p$.
        \item $p\inv\concat p=\refl$, $p\concat p\inv=\refl$.
        \item ${(p\inv)}\inv=p$.
        \item $p\concat(q\concat r) = (p\concat q)\concat r$
    \end{enumerate}
\end{lemma}

Analogous to the fundamental group, we can define the {\it loop space} 
$$\Omega(A, a)\defn a=_Aa$$ for a type $A$ and $a: A$. More precisely,

\newcommand{\pointed}{_\bullet}
\begin{definition}
    We call $(A, a)$ a {\it pointed type} for $A:\universe$, $a:A$.
    $a$ is called the {\it basepoint} and the pair of type 
    $\universe\pointed\defn\sum_{(A:\universe)A}$ is the type of pointed 
    types in universe $\universe$.
\end{definition}

\begin{definition}
    Given any pair $(A, a)$, we define the {\it loop space} as 
    $$ \Omega(A,a)\defn ((a=_A a),\refl_a).$$
    For $n:\nat$, we define the {\it$n$-fold iterated loop space} as
    $$\Omega^0(A,a)\defn(A,a)$$
    $$\Omega^{n+1}(A,a)\defn\Omega^n(\Omega(A, a))$$.
\end{definition}

For example, 
$$
    \Omega^2(A,a)\equiv\Omega(\Omega((A,a)))\equiv\Omega(a=_Aa,\refl_a)
    \equiv(\refl_a=_{a=_Aa}\refl_a, \refl_{\refl_a}).
$$

Just like ``{\it Groups in the category of groups are commutative}''
\cite{cat-awodey} or higher homotopy groups are commutative, we have
the Eckmann-Hilton theorem in type theory namely that the concatenation of 
paths $\Omega^2(A,a)\times\Omega^2(A,a)\to\Omega^2(A,a)$ is commutative.


\section{Functoriality and Fibration}

Now, it's reasonable to think of a type as a groupoid (category).
The Leibniz rule (\autoref{leibniz}) shows a kind of functoriality.

\newcommand{\ap}{\name{ap}}
\begin{lemma}
    For a function $f: A\to B$ and $p: x=_A y$, then we have
    $$
        \ap_f: (x=_A y)\to (f(x) =_B f(y)).
    $$
    Furthermore, given $g: B\to C$ and $q: y=_A z$, we have
    \begin{enumerate}
        \item $\ap_f(p\concat q)=\ap_f(p)\concat\ap_f(q)$
        \item $\ap_{\id_A}(p)=p$
        \item $\ap_f(p\inv)=\ap_f(p)\inv$
        \item $\ap_g(\ap_f(p))=\ap_{g\circ f}(p)$
    \end{enumerate}
\end{lemma}

\begin{proof}
    (In a category, the name $\ap$ is preferred over {\it Leibniz rule}.)
    To prove it, we first define a type family $D(x,y,p)\defn f(x)=_Bf(y)$.
    According to the induction rule. To define a function out of the type
    $x=_Ay$, it suffices to inhabit the type $\Pi_{x:A}D(x,x,\refl_x)$.
    Note $D(x,x,\refl_x)=f(x)=_Bf(x)$. We have
    $$
        d\defn \lambda x.\refl_{f(x)}:\Pi_{x:A}D(x,x,\refl_x).
    $$
    Hence we have proved
    $$
        \ap_f(p)\defn \ind_{=_A}(D,d,x,y,p): D(x,y,p).
    $$
    By the computation rule, we have 
    $$
        \ap_f(\refl_x) \equiv \refl_{f(x)}.
    $$
    The rest are similar.
\end{proof}

Sometimes we just write $f(p)$ for $\ap_f(p)$ for convenience. Thus
we can draw the commutative diagrams for the above lemma.

$$
\begin{tikzcd}
    A \arrow{r}{\id_A} & A \arrow{rr}{f} && B \arrow{rr}{g} && C
    \\
    x \arrow[dd, rightsquigarrow, bend right, "p"{name=p1, left}] & 
    x \arrow[dd, rightsquigarrow, bend right, "\id_A(p)"{name=idp, left}]
    \arrow[dd, rightsquigarrow, "p"{name=p, right}]
    \arrow[rd, rightsquigarrow, "p\concat q"]
    \arrow[Rightarrow, from=idp, to=p]
    & &
    f(x) \arrow[dd, rightsquigarrow, "f(p)"{left}]
    \arrow[rd, rightsquigarrow, "f(p\concat q)"{name=fpq,left}]
    \arrow[rd, rightsquigarrow, bend left, "f(p)\concat f(q)"{name=fpfq,right}]
    \arrow[Rightarrow, from=fpq, to=fpfq]
    & & g(f(x)) 
    \arrow[dd, rightsquigarrow, "g(f(p))"{name=gfp1, left}]
    \arrow[dd, rightsquigarrow, bend left, "(g\circ f)(p)"{name=gfp2, right}]
    \arrow[Rightarrow, from=gfp1, to=gfp2]
    \\
    & & z & & f(z) \\
    y & y \arrow[ru, rightsquigarrow, "q"] & & f(y) \arrow[ru, rightsquigarrow, "f(q)"] & & g(f(y))
\end{tikzcd}
$$

It is natural to think about the functoriality of dependent functions.
This is not as simple as its non-dependent version, because for an
$f: \Pi_{x:A}B(x)$ and $p: x=_A y$, $f(x): B(x)$ and $f(y):B(y)$ lie in
different types. This difficulty raises the fibration of a type.

The reader may be familiar with tangent bundle or fibre bundle and may
have already noticed that formally, such a bundle is a dependent pair.
For a type family $P: A\to \universe$, we may consider $P(a): \universe$ as the fibre
over $a: A$. Thus $\Sigma_{a:A}P(a)$ is the bundle gathering all the
fibres. Besides, the projection $\pi_1: \Sigma_{a:A}P(a)\to A$ acts as
the bundle projection, which suggests that we should call 
$\Sigma_{a:A}P(a)$ the {\bf total space} and $A$ the {\bf base space}.
We also call $P:A\to\universe$ the fibration directly. 

\newcommand{\transport}{\name{transport}}
Thus to compare two elements in different fibres, we can analogously
define the {\it transport} between them.
\begin{lemma}\label{transport}
    For $P: A\to\universe$ and $p: x=_Ay$, we have $p_\ast: P(x)\to P(y)$.
    Sometimes if we need to indicate thhe $P$ explicitly, we write
    it as $\transport^P(p,-):P(x)\to P(y)$
\end{lemma}
\begin{proof}
    As previously, let $D(x,y,p)\defn P(x)\to P(y)$. We only have to prove
    $D(x,x,\refl_x)\equiv P(x)\to P(x)$ for all $x:A$, which is certainly
    inhabited by $\id_{P(x)}$.
\end{proof}
$$
\begin{tikzcd}
    P(x) \arrow[rrr, "p_\ast"] & & & P(y) \\
    & x \arrow[lu, bend right, dash] \arrow[r,rightsquigarrow,"p"]& 
    y \arrow[ru, bend left, dash] &
\end{tikzcd}
$$

\begin{lemma}
    Here are some properties of $\transport$.
    \begin{itemize}
        \item For $P: A\to\universe$, $p: x=y$, $q: y=z$, $u: P(x)$, we have $q_\ast(p_\ast(u))=(p\concat q)_\ast(u)$.
        \item For $f: A\to B$, $P:B\to\universe$,$p:x=y$, $u: P(f(x))$, we have $\transport^{P\circ f}(p,u)=\transport^P(\ap_f(p), u)$. 
        \item For $P,Q: A\to\universe$, $f:\Pi_{x:A}P(x)\to Q(x)$, $p:x=y$, $u: P(x)$, we have $\transport^Q(p,f_x(u))=f_y(\transport^P(p, u))$. 
    \end{itemize}
\end{lemma}

\newcommand{\lift}{\name{lift}}
Now we can handle the dependent version of functoriality. First let's
discuss the lift of a path in the total space. 
\begin{lemma}\label{lift}
    For a fibration $P: A\to U$ and a path $p: x=y$ in the base space,
    given a starting point $u:P(x)$, we can define a lift
    $$
        \lift(u, p): (x,u)=(y, p_\ast(u))
    $$
    in the total space $\Sigma_{x:A}P(x)$ such that $\pi_1(\lift(u, p))=p$.
\end{lemma}

\begin{remark}
    In $\Sigma_{x:A}P(x)$, we should think $(x, u)$ as the starting
    point, but since $u: P(x)$, giving $u$ is equivalent to giving the
    tuple $(x, u)$. $\lift(u, p): (x,u)=(y, p_\ast(u))$ is a path in
    the total space. According to the functoriality, 
    $\pi_1: \Sigma_{x:A}P(x)\to A$ can be extended to
    $\ap_{\pi_1}: (=_{\Sigma_{x:A}P(x)\to A})\to(=_{A})$, so
    $\pi_1(\lift(u, p))$ is actually $\ap_{\pi_1}(\lift(u, p)):\pi_1(x,u)=\pi_1(y,p_*(u))$.
\end{remark}

$$
\begin{tikzcd}
    \Sigma_{x:A}P(x) \arrow[d,"\pi_1"] & 
    (x,u)
        \arrow[rd, mapsto, "\pi_1"] 
        \arrow[rrr, rightsquigarrow, "\lift(u{,}p)"]
    & & & 
    (y, p_\ast(u)) \arrow[ld, mapsto, "\pi_1"] 
    \\
    A & & x \arrow[r, rightsquigarrow, "p"] & y &
\end{tikzcd}
$$

\begin{proof}
    By setting $D(x,y,p)\defn(x,u)=(y,p_\ast(u))$
\end{proof}

Second, note that by setting $f'(x)\defn (x, f(x)): A\to\Sigma_{x:A}P(x)$, we 
can apply the functoriality for $f: \Pi_{x:A}P(x)$ as in the following 
diagram, where $f(x)$ means $(x, f(x))$ and $f(p)$ is $\ap_{f'}(p)$. 

$$
\begin{tikzcd}
    \Sigma_{x:A}P(x) \arrow[d,"\pi_1"] & 
    f(x)
        \arrow[rd, dash, bend left, "P(x)"{left}] 
        \arrow[rrr, rightsquigarrow, "f(p)"]
    & & & 
    f(y) \arrow[ld, dash, bend right, "P(y)"{right}] 
    \\
    A & & x \arrow[r, rightsquigarrow, "p"] & y &
\end{tikzcd}
$$

In this solution, we abandoned the information about the $p:x=y$, which may
be important sometimes. Since we also have a $\lift(f(x), p): (x, f(x))=(y,p_\ast(f(y)))$,
it is very natural for us to think about the equality $(y, f(y))=(y,p_\ast(f(y)))$.

\newcommand{\apd}{\name{apd}}
\begin{lemma}
    For $f:\Pi_{a:A}P(x)$, we have
    $$
    \apd_f: \prod_{p:x=y}p_\ast(f(x))=_{P(y)}f(y)
    $$
\end{lemma}
\begin{proof}
    Let $D(x,y,p)\defn p_\ast(f(x))= f(y)$.
\end{proof}

More generally, for $u: P(x)$ and $v: P(y)$, we call the equality
$p_\ast(u)=_{P(y)}v$ ``{\it a path from $u$ to $v$ lying over $p$}''\footnote{This comes from \cite{homotopy-type-theory}}.

\section{Homtopies in type theory}

The behavior of the type theory quite resembles homotopy in algebraic
topology. Let me give more notations.

In set theory, two functions $f,g: A\to B$ are equal iff 
$f(x)=g(x)\forall x\in A$ as ``relations''. In type theory this is wrong
because there's no evidence that $f: A\to B=g: A\to B$ ensures they are
defined with the same lambda term (with respect to $=_\beta$). For example 
$\lambda x:\nat. x+0$ and $\lambda x:\nat. x$ are definitionally different.
(Note that $x+0=x$ is proved not $=_\beta$). As proved before, we can
prove $\prod_{n:\nat}(\lambda x.x+0)\ n=(\lambda x.x)\ n$. You can even
verify this in pure $\lambda$-calculus (recall that natural numbers are
represented by church numerals). 
The axiom\footnote{This is not included by the theroy presented so far.}
$(\Pi_{x:A}f(x)=g(x))\to f = g$ is known as {\it function extensionality},
which will trivialize the intensional homotopy structure of the theory.
Thus in HoTT, this is viewed as a homotopy between $f$ and $g$.
\begin{definition}
    For $f,g:\Pi_{x:A}P(x)$, the homotopy relation between $f$ and $g$
    is a dependent function
    $$
        f\sim g\defn \prod_{x:A}f(x)=g(x)
    $$
    ($f$ is homotopic to $g$).
\end{definition}

\newcommand{\isequiv}{\name{isequiv}}
Similarly, we can define the homotopy equivalence $\simeq$.
\begin{definition}
    If $f:A\to B$ is a function, we say that $f$ is an equivalence provided
    $\exists g:B\to A,f\circ g\sim \id_B$ and 
    $\exists h:B\to A(h\circ f\sim \id_A)$, i.e., 
    $$
        \isequiv(f)\defn(\sum_{g:B\to A}f\circ g\sim \id_B)\times
        (\sum_{h:B\to A}h\circ f\sim \id_A)
    $$
    And we can define
    $$
        A\simeq B\defn\sum_{f:A\to B}\isequiv(f)
    $$
\end{definition}

Note we donot define it as $\sum_{g:B\to A}(f\circ g\sim id_B)\times
(g\circ f\sim \id_A)$, where $g$ is called {\it quasi-inverse} of $f$.
One problem is that the inhabitants of that type is not unique 
(propositionally), and it can be shown that they are `equivalent' (two 
types imply each other), whose proof is similar to that in algebraic 
topology (a problem in the mid-term last semester). 

Now think about the path $=_A$ in the space $\universe$. We haven't 
inspected any further information about the universe $\universe$.
With the notation of equivalence, Voevodsky gave an extension of the
equality as an axiom (called {\it Univalence Axiom})
\newcommand{\univalence}{\name{univalence}}
$$
    \univalence: (A=B)\simeq(A\simeq B).
$$
You can easily find the canonical function $\Id_\universe(A,B)\to(A\simeq B)$. 
The univalence axiom extends it into an equivalence. It is said that this
axiom raises many consequences ``{\it naturally, simply and compellingly}''
\cite{homotopy-type-theory}. 

For example when defining a homotopical circle $\sphere$, we can define 
it as a base point and a loop (path, i.e. the equality) between the 
basepoint
\newcommand{\base}{\name{base}}
\newcommand{\nloop}{\name{loop}}
$$
    \sphere\defn\begin{cases}
        \base: \sphere \\
        \nloop: \base=\base \\
    \end{cases}.
$$
This kind of style to define a new type (inductively) is called higher
inductive type (HIT) (chapter 6 of \cite{homotopy-type-theory}).
To make a function $\sphere\to A$, we have to find an $a:A$ and
a $p: a=a$, i.e. $\rec(a, p): \sphere\to A$ such that 
$\rec(a,p)(\base)\defn a$ and $\rec(a,p)(\nloop)=p$. 

To calculate its fundamental group, we find the universal covering space 
of it, namely a family $\name{cov}:\sphere\to\universe$. That is to
find an $A: \universe$ and an equality $A=_\universe A$. Letting $A$
be $\integer$, we define $\name{cov}(\base)\defn\integer$ and 
$\name{cov}(\nloop):\base=\base$. According to the univalence axioms,
it suffices to define an equivalence $\integer\simeq\integer$. In fact
we know that $\succn: \integer\to\integer$ is clearly what we want.

The space $\sum_{x:\sphere}\name{cov}(x)$ forms a covering space of
the homotopical sphere $\sphere$. For any path $p:x=_\sphere y$,
we have a {\it transport} $p_\ast:\name{cov}(x)\to \name{cov}(y)$.
Thus we define the $\name{wind}$
function $\name{wind}: \Omega(\sphere,\base)\to\integer$ such that
for any $p:\base=\base$, $\name{wind}(p)=p_\ast(0)$, which is the
inverse of $n\mapsto \nloop^n: \integer\to\Omega(\sphere)$.

It's quite natural to define the circle $\sphere$ with this homotopy
interpretation. I think it's easier to describe the higher category,
e.g. the fundamental $\infty$-groupoid with the language of HoTT. 
Besides, it seems that this language has some other applications.
In a talk given by Andr\'e Joyal at MIT \cite{joyal2014categorical}, 
it is suggested that the theory can be further applied to higher topos 
theory or higher category theory.
