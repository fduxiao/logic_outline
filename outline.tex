\documentclass[12pt,a4paper]{report}

\usepackage{proof}
\usepackage{amsmath,amssymb,amsthm}
\usepackage{enumitem}
\usepackage{stmaryrd} 

\theoremstyle{definition}
\newtheorem{definition}{Definition}[chapter]
\newcommand*{\definitionautorefname}{Definition}

\theoremstyle{plain}
\newtheorem{theorem}{Theorem}[chapter]

\theoremstyle{remark}
\newtheorem*{caveat}{\it CAVEAT}
\newtheorem{remark}{\it Remark}[chapter]

\newcommand{\eq}[1]{\begin{align*}#1\end{align*}}

\newcommand{\inv}{^{\raisebox{.2ex}{$\scriptscriptstyle-1$}}}
\newcommand{\universe}{\mathcal{U}}
\newcommand{\nat}{\mathbb{N}}
\newcommand{\zero}{\mathbf{0}}
\newcommand{\one}{\mathbf{1}}
\newcommand{\two}{\mathbf{2}}
\newcommand{\defn}{:\equiv}
\newcommand{\name}[1]{\text{\sf #1}}
\newcommand{\proves}{\vdash}
\newcommand{\iu}{{i\mkern1mu}}
\newcommand{\complex}{\mathbb{C}}

\usepackage{mathrsfs}
\usepackage{titlesec}
\usepackage{hyperref}
\hypersetup{
    colorlinks=true,
    linktoc=all,
    linkcolor=blue,
}

\usepackage[english]{babel}
\addto\extrasenglish{%
  \def\chapterautorefname{Chapter}%
}

\begin{document}

\title{A Summary of Logic}
\author{Xiao}
\date{}
\maketitle

\tableofcontents

\chapter*{Introduction}
\addcontentsline{toc}{chapter}{Introduction} 

Several years ago, when I was programming, I was fascinated
by the concept of meta-programming, i.e., generating code with code
so that I can write less code. Generally, you write some code for
a real problem. Even if your target is to generate code, you have
to execute the code to get the result. This sounds nonsense to a 
mathematician, but this is actually how we handle mathematical 
logic, i.e., we define our language as mathematical objects 
(We define the well formed formula or proof as sets) 
just as we define a group with set theory. When we want to prove
something, we follow the predefined syntax (e.g. a deduction system)
constructing a proof to consolidate a belief. We never infer anything
about this syntax itself once it is defined. This is because if we
want to infer about the inference itself, we may end up with the 
Rousell's paradox. However, we study the structure of reasoning
safely as a mathematical object. Do we still infer about the inference
so that it might trigger the paradox? The basic observation is that 
we treat it as a {\bf standalone} object, an identical but distinct 
object from the meta one we use. This is also adopted by 
computer scientists, i.e., they can write a program (usually called 
a compiler) that turns or generates the source code (as objects) 
written in the same or different language to a program. Typically, 
the language they use to write the program is called the meta language, 
while the outcome language is the target language. They are generally 
independent. One is analysized as objects of the other. 

But why do I have to emphasize programming language? A shocking fact is 
that defining inference rule and applying it to another inference is 
widely and naturally accepted in programming. For example, you may have
proven that to prove $\phi$ or $\psi$, you only need to prove $\psi$ 
assuming $\phi$ is wrong. I used this in the proof of 
{\it The boundary of a union is contained in the union of boundaries} 
($\partial(A\cup B)\subseteq \partial A \cup\partial B$). 
When proving that proposition in {\it Topology Theory}, I can never 
write a sentence like $\forall A,B\in Props,\phi\vee\psi \leftrightarrow
\neg\phi\rightarrow\psi$, because the Topology Theory knows nothing
about the concept $Props$. The proof of it is outside the theory,
i.e., in the second-order language, but in our language, we can
prove both. When it comes to the programming language, you can also
prove them in the same context. According to the Curry-Howard 
Correspondence \cite{Curry-Howard,Curry-Howard-Scheme}, such a
proposition $(\neg\phi\to\psi)\to\phi\vee\psi$ is understood as a 
function, which, given sets \footnote{For an intuitive illustration, 
here I use set, but it later turns out a type.} $\phi$ and $\psi$, 
gives you a function mapping any function from 
$\neg\phi\to\psi$ to the coproduct $\phi\vee\psi$. (This is 
called a dependent type, showing a kind of quantification. See 
\autoref{type-theory} for a formal description.) When using it to
prove the propsition about boundary($\partial$), you apply this
function to some $\phi$, $\psi$ and $\neg\phi\to\psi$ to generate
a $\phi\vee\psi$. Certainly, this is not a set-theoretic function,
since $\phi$ and $\psi$ traverse the whole universe of sets. Actually,
the behavior is like the $\forall$ quantification, which does traverse
all sets. When applying it to some element, the compiler will determine
the suitable set $\phi$ and $\psi$, which won't cause you any trouble
except that you want to apply it to itself, i.e., one can still lift
the order of the language or he/she gets the paradox. A clever solution 
towards this is building a hierarchical universe as von Neumann did 
\cite{von-Neumann-universe}. Once $\phi$ and $\psi$ are picked, the 
compiler will infer the order of our language. 

In our real world, this is what mathematicians want, an any order
language or a semantic language. For example, if we take the 
first-order language to be the language of group theory $(\cdot,e)$, 
then the (structured) set theory (model theory) is the second order one.
A proof about group properties can be done with merely group theory,
but do you really use it alone? As in any textbook, the author always
takes a group as a set and apply set theory to prove, and properties
beyond group theory (like morhpisms) are also discussed with set
theory. For a third order example, logical structures are also proved
somewhere. Recall that you must have encountered some problem in
group theory or set theory where you want to use a tautology to turn 
it into a logically equivalent problem but you have to check whether 
this tautology is correct or not. Some may argue that they can certainly 
introduce all these tautologies
into the deduction system (via schemes\footnote{A collection of
axioms. Scheme is also the name of a programming language}), so you 
only have to deal with the first order one \footnote{Unlike the behavior 
discussed above, a C++ compiler only accepts such functions with all 
quantifications at the beginning like schemes, i.e. the quantifier 
prenex form and only universal quantifications are accepted, while 
Haskell has a language extension to allow existential quantifications 
so that you can put the quantification at the middle of a sentence.
Idris, another programming language, adopts the hierarchical universe
with orders of the language. The macro system of C is a totally 
separation from the target language as a meta language. They coincide
with our handling in logic.} (the set theory is also flattened with
group theory as both are first order), but as you see, in your real 
practice, you'd like to verify it first and then apply it. (This means
the inference rules should be finite.) Without always 
explicitly specifying the order of the language, one can ignore this 
problem to some extent. A clever computer proof-assistant system will 
help you with it, or one can just use his/her brain to tell the order. 
Once this order is determined, the rest is a plain set theory problem 
first-orderly. (Note also that the set theory used during the 
development of first order language is different from the one developed 
by first order language later in the fact that they are in different
universe with different orders). Another notable part is that the
properties of a group (functions and equality) coincide with those of 
set (definable with set theory). Thus we can represent a group with a
set. From this perspective, group theory is a domain specific language
\footnote{This concept is also borrowed from progromming language.
In real world, a programmer only focus on the domain of his/her task.
They want to represent a real world problem with turing machine. Idris
provides a syntax suger to freely define any domain specific gramma.} 
(DSL) represented by set theory, or you can call set theory the frontend 
\footnote{The direct interface we write down in our books or on paper. 
In programming world, Tensorflow or PyTorch use python as their frontend 
language.} language for group theory, i.e. every set operation (only the 
equality comparison and binary operation) is mapped to an operation in 
group. 

To summary, we hope a self reflexive and representative language. The
first order language together with sets (models) as its semantics is
representative but it is not good at self referring (You have to lift
the order of the language and treat it separately), and this is hard
for a computer assistant to deal with. As a substitution, type theory
embrace these benefits. The type Theory \cite{intuitionistic-type-theory} 
chooses {\it type} as its formal language, which, just as the model 
(structured set) theory chooses set theory for its semantics, can choose 
(the hierarchical) {\it Topos} as its semantics 
\cite{Introduction-to-higher-order-categorical-logic,
categorical-logic-and-type-theory}. Later the Homotopy Type Theory(HoTT)
\cite{homotopy-type-theory,univalence} also gives another 
geometric interpretation for type theory. The benefit of type
theory is its intrinsic computational nature. Inference rules
can be verify by a computer, a proof is constructive, 
and the type hints a mathematician about his/her goal\cite{coq}.

The computational explanation for logic also raises other problems.
In general we can not compare two proofs. One can always write down
redundant sentences amidst a proof, as is a redundant calculation
abstraction. To find the shortest form of a calculation is known as
the normalization problem in typed lambda calculus\cite{Curry-Howard}.
Again, the Curry Howard correspondence provides us a way to compare two
proofs. The meta-programming techniques, i.e generating some parts of
the proof from the other, can also be applied to proving. A quick example
is quantification elimination, which provides an algorithm to transfer
each proof of a quantification-free formula to an equivalent one with
quantifications under certain circumstances. Haskell
turns its language into a category and uses monad to represent
complicated computation process 
\cite{the-essence-of-functional-programming}. 
In \autoref{intuitionistic-logic},
I will give a monadic proof for Glivenko's theorem, and make use of
it as a monad in \autoref{monad-algebra}. A combinatory style logic 
based on the continuation monad is also proposed born with computation 
in its nature. 

Based on these observations, a kind of constructive, categorical and 
thus algebraic and combinatory logic is needed for mathematics. The
syntactic part is developed with category theory and type theory in
computer science, but most computer scientists care only about the
syntax itself, ignoring its potential application in mathematics. 
Recently, the HoTT rises for a explanation (semantics)
of type theory in mathematics. It seems that topos theory is an
explanation of it geometrically and the two theories also seem related.
For my future, I hope I can understand the theories and fill in some
blanks so that I have enough tools to learn mathematics with type
theory. I hope they provide me an instrument to make my proof shorter
and concepts easy to remember (It's easier to remember the Jordan
canonical form with module theory then pure linear algebraically).

My current difficulties lie in the understanding of $\infty$-groupoid
used as a basic structure for HoTT. A lack of the whole
picture also stops me from getting started. I've found a lot of 
similarities among them, but I haven't found a universal explanation. 
In this outline, I will briefly describe what I've observed. 
\autoref{type-theory} defines the basic language used for the
following and it is then related to logic in 
\autoref{intuitionistic-logic}. The usage of monad and the further
raise of combinatory logic are handled in \autoref{monad-algebra} and
\autoref{combinatory-logic}. I also provide \autoref{monad-examples}
for other examples of monad. The Homotopy Type Theory is discussed
in \autoref{HoTT}. 
\chapter{Type Theory}
\label{type-theory}

In this chapter, I mainly describe a basic language like set theory.
The semantics of first-order language is strongly connected to sets.
Every first-order formula represents a kind of set operation or property.
For example, we can prove a group of order prime squared is abelian.
Let me first formalize it in first order logic. The language should
be $\mathscr{L}=(\ast)$, and the axioms are
\begin{align*}
    \Gamma & = \{ & \\
        & \forall a b c\ a\ast(b\ast c)=(a\ast b)\ast c,\\
        & \exists e(\forall a\ e\ast a=a\ast e = a)
            \wedge (\forall a\exists a\inv(a\inv
                \ast a = a\ast a\inv=e))\\ 
    \}.
\end{align*}
The semantics behind these first order sentences is a set $G$,
together with an extra operation $\cdot: G\to G$. The first 
axiom says some property of this operation, while the second
can be interpreted in two steps. Half of it, $\exists e
(\forall a\ e\ast a=a\ast e = a)$\footnote{$\Gamma\models
\exists e(\forall a\ e\ast a=a\ast e = a)$.}, asserts the
existence of an element $e^G\in G$ such that for any element
$a\in G$, $e^G\cdot a= a\cdot e^G=a$, and it is also easy to
show this $e^G$ is unique in $G$. The other half claims every
element is invertible. The language itself means operations in
sets, i.e. the logic is split into two parts, the deduction rules and
the set-theoretical semantics of it\footnote{The HoTT book
\cite{homotopy-type-theory} gives a detailed description.}. When
you are proving a proposition (well-formed formula), you follow the
deduction rules to obtain a correct proof, while from the semantic
side, each sentence you write gives you a judgement (equality or 
existence) about the underlying set of the group. Now I want to say
it has prime squared order $5^2$, thus adding to $\Gamma$ the axiom
$\exists x_1\exists x_2\cdots\exists x_{25}, x_1\neq x_2\wedge\cdots
\wedge x_{24}\neq x_{25}\wedge\forall y, y=x_1\vee\cdots\vee y=x_{25}$. 
Since semantically we know it's true that any such a group(model) is 
abelian, then there must be a syntactic proof of the sentence 
$\forall x\forall y\ x\ast y=y\ast x$ (completeness).

I don't konw what exactly the proof is (you may want to try proving it 
with pure first order logic), but let me show what's the semantic proof. 
Normally, the proof of the abelianity is that for any group $G$ of order 
$p^2$, where $p$ is a prime, we can pick the center 
$Z =  \{g\in G\mid\forall a\in G\ ag=ga\}$ of it. Thus it can only has 
order $1$ or $p$ or $p^2$. According to the class equation, 
$|G| = |Z| + \sum_iC_i$, where $C_i$ are conjugation classes. Thus $|Z|$
can only be $p$ or $p^2$. If $|Z|\neq p^2$, pick an $x\in G\setminus Z$
The centralizer $Z(x) = \{g\in g\mid gx=xg\}$ must have order $p^2$, i.e.
$x$ should be in $Z$, which is a contradiction.

In general, we use a semantic proof everywhere. From the perspective of
first order language, we define how the set (model) structure is formed,
while in real world, we perform the computation (of sets) directly. If
you know how the computation is made, you must be able to write down the
proof (though it may be very difficult, as the example above). Now think
what are the first order language and its inference rules. In mathematical 
logic, the langauge itself is defined as object (set), and the rules are
defined as computations (some kind of functions). (Note both the 
{\it modus ponens} and axioms represent whether you are allowed to 
introduce or eliminate certain sentences. Just think them as buttons on
a calculator.) Again, the axioms in ZFC (when ZFC is considered as the
identical but distinct version of the one we use in meta langauge) are 
also such rules. In summary, the logic language is the object, our
inference is doing computations just like applying a function to an
element deriving another. 

This coincidence (the inference rules can be viewed as 
object operations ) is called {\it Curry-Howard 
Correspondence}. But instead of set theory, typed lambda calculus
\cite{Curry-Howard} is chosen for it, mainly because it's more natural 
in computer programming \cite{coq,SF}, and some syntactic rule of lambda 
calculus coincides with modus ponens. So you won't have to treat the 
first order language specially, or you may get Rousell's paradox
(one layer of deduction system and syntax and another layer of semantics). 
Look at the group above, the semantic proof is very natural in our 
language, but is not applicable pure syntactically. In the rest of this 
chapter I'm going to introduce the type theory (mainly a re-illustration 
of the first chapter of \cite{homotopy-type-theory}). Bear in mind that 
everything is associated with a construction and under such a scheme, 
no extra relation between the first order layer and the semantic layer
is needed. 

\section{Lambda-Calculus}
The core of our language is the lambda calculus. I first start with a
untyped version of them, and then give a simply-typed version. You don't
have to think about its real meaning. Just play with it until you are
happy with this concept. (See chapter 1, 3 of \cite{Curry-Howard})

Let 
$V=\{v_0, v_1, \cdots\}$ be a variable set. Our objects, the 
$\lambda$-$terms$, are defined as follows\footnote{This notation is
called the backus normal form}. 

\begin{definition}\label{lambda-term}
    $$\Lambda ::= V \mid (\Lambda\ \Lambda) \mid (\lambda V\ \Lambda)$$
\end{definition}

You can understand it as:
\begin{enumerate}
    \item A variable is a $\lambda$-$term$.
    \item If $T_1$ and $T_2$ are $\lambda$-$terms$, then so is $(T_1\ T_2)$.
    This is called an application.
    \item If $x\in V$ is a variable, $T$ is a $\lambda$-$term$, then so
    is $(\lambda x\ T)$. This is called an abstraction. 
\end{enumerate}

The abstraction is something like $f(x) = x^2 + 1$. Equivalently, you
can define it as $f=\lambda x\ (x^2+1)$, though $2,+,1$ are not defined.
As you may guess, the application is purely substitution.

\begin{definition}\label{beta-reduction}
    We define a relation $\to_\beta$ between $\lambda$-terms
    $$((\lambda x\ P) Q) \to_\beta P[x:=Q],$$ where $P[x:=Q]$ means the 
    substitution of $Q$ for every occurence of $x$ in $P$. 
\end{definition}

For example $((\lambda x\ (x+x))\ 2)\to_\beta(x+x) = 4$, though
this is certainly not well defined. I believe you can figure out what
I mean. This means a step of calculation, called $\beta$-reduction.
In \cite{homotopy-type-theory}, this is the computation rule for function
type. The relation $\to_\beta$ is defined for the application of an 
abstraction and another term. You may want its transitive closure 
$\twoheadrightarrow_\beta$ (multiple steps of $\beta$-reduction) or 
reflexive closure $=_\beta$ (they can be both reduced to the same
$\lambda$-term).

As you may guess again, there's also the annoying problem about free
variables. Let me first make some conventions for convenience (to
avoid too many parentheses). 
\begin{enumerate}
    \item $(\lambda x\ \lambda\ y\ T)$ is $(\lambda x\ (\lambda y\ T))$.
    This is called {\it currying}, i.e., a function $f: X\times Y\to Z$ is
    a function mapping each $x\in X$ to a function $f(x): Y\to Z$ 
    (Adjunction of the category $Set$)
    \item In response, $(K\ L\ M)$ is $((K\ L)\ M)$, just like $f(x, y)$
    is $f(x)(y)$ (Note $f(x): Y\to Z$ is a function). 
    \item $(\lambda x\ M\ N)$ is $(\lambda x\ (M\ N))$.
    \item $(M\ \lambda x\ N)$ is $(M\ (\lambda x\ N))$.
\end{enumerate}

The outermost parentheses are also omitted if that won't cause ambiguity.
We also write $\lambda x_1 x_2\cdots x_n.M$ for 
$(\lambda x_1\ (\lambda x_2\ (\cdots(\lambda x_n\ M))))$. Another notable
observation is that the abstraction is independent of the varible used.
For example $\lambda x.x$ and $\lambda y.y$ are identical. You can freely
change the variable used, but be careful about the `naming conflict'. 
The term $\lambda x. \lambda x. x\ x$ is understood as 
$\lambda y. \lambda x. x\ x$. (The renaming process is called {\it
$\alpha$-conversion}. )

For the real world, you can introduce numbers and basic arithematics like
the example $\lambda x. x \cdot 2 + 1$, but for a pure theoretical 
purpose, you can define the numbers as $c_n=\lambda s.\lambda z. s^n\ z$, 
where $s^n\ z = s\ (s\ (\cdots\ (s\ z)))$. Those $c_n$ are called {\it
Church numeral}. Try to prove the term $A_+=\lambda x.\lambda y.\lambda s
.\lambda z.x\ s\ (y\ s\ z)$ satisfies the relation 
$A_+\ c_n\ c_m=_\beta c_{m+n}$. Consider what are the multiplication and 
exponent. The last example is only available in untyped lambda calculus,
the fixed point combinator. Compared with the traditional 
\newcommand{\Ycomb}{\mathbf{Y}}
$$\Ycomb = \lambda f.(\lambda x.f\ (x\ x))\ \lambda x.f\ (x\ x)$$
such that for any $\lambda$-term $F$, 
$F\ (\Ycomb)=_\beta \Ycomb\ F$, the exercise in chapter 1 of 
\cite{Curry-Howard} is more interesting. Let
$$
\begin{array}{lll}
    ? &=& \lambda abcdefghijklmnopqstuvwxyzr.r\ (thisisafixedpointcombinator) \\
    \$ &=& ??????????????????????????
\end{array}
$$
Prove that $\$\ F=_\beta F\ (\$\ F)$ for all $F\in\Lambda$.

Then let's talk about typed lambda calculus. All the terms we defined
above can be applied to any other term, but for a function 
$f: \mathbb{N}\to\mathbb{N}$, you won't apply it to $\sqrt 2$ and 
you won't expect the result is an element of the set $\{cat,dog,duck\}$.
Thus we attach each $\lambda$-term with such an annotation. 

Let $U$ be the set of all {\it type variables}. Define the set 
$\Delta$\footnote{In \cite{Curry-Howard}, the letter for such a set is
$\Pi$, but this is used for dependent function type ($\Pi$-type) 
(\autoref{dependent-function}) in HoTT, so I change the notation.} as
$$
    \Delta ::= U \mid \Delta \to \Delta
$$
Now the $\lambda$-terms occur with a type in $\Delta$. For example,
$\lambda x. x: a\to a$ for some $a\in U$. Formally speaking, the
notation `:' is introduced as follows. 

A context is such a set $\Gamma = \{x_1: \tau_1, \cdots, x_n:\tau_n\}$ 
with $x_i\in U$ and $\tau_i\in V$. ($U$, $V$ are variables for 
$\lambda$-terms and types respectively.) We define the {\it typability}
relation $\proves$ on the set of all triples $(\Gamma, x, \tau)$, where 
$x\in\Lambda$ and $\tau\in\Delta$ (written as $\Gamma\proves x:\tau$).

\begin{enumerate}
    \item $$\infer{\Gamma, x:\tau\proves x:\tau}{}$$
    This means without any premise (the blank on the line), you
    are allowed to define $\proves$ for triple 
    $(\Gamma\cup\{x:\tau\}, x, \tau)$
    ($\Gamma,x:\tau$ means $\Gamma\cup\{x:\tau\}$).
    \item $$\infer{\Gamma\proves\lambda x.M:\sigma\to\tau}{\Gamma,x:\sigma\proves M:\tau}$$
    This means if you have defined the relation $\Gamma,x:\sigma\proves M:\tau$,
    then you can obtain a new relation $\Gamma\proves\lambda x.M:\sigma\to\tau$.
    (introduction rule for $\to$)
    \item $$\infer{\Gamma\proves M\ N:\tau}{\Gamma\proves M:\sigma\to\tau & \Gamma\proves N:\sigma}$$
    This means you can eliminate the arrow $\to$ in the type by application.
\end{enumerate}

Note we start with the context $\Gamma$, where only variables are annotated
by types. Our target is to define how the $\lambda$-terms are annotated.
Thus a suitable target is to introduce all $\to$ until $\Gamma$ is empty
(rule 2). For example
\begin{enumerate}
    \item $\proves \lambda x.x: \sigma\to\sigma$
    \item $\proves \lambda x.\lambda y.x: \sigma\to\tau\to\sigma$ \\
    (according to currying, $\sigma\to\tau\to\sigma$ should be 
    $\sigma\to(\tau\to\sigma)$)
    \item $\proves \lambda f.\lambda g.\lambda a. f\ a\ (g\ a):
    (A\to(B\to C))\to(A\to B)\to(A\to C)$
\end{enumerate}

Coincidently, the context can be given another interpretation. The 
{\it range} of a context $\Gamma$ is defined by
$|\Gamma|=\{\tau\in\Delta\mid(x:\tau)\in\Gamma\}$. The type set $\Delta$
can also be understood as the set of all well formed formulas (without
conjunction or disjunction) in propositional logic. Then you can easily 
observe the following theorem.
\begin{theorem}
    (Curry-Howard Correspondence).
    \begin{enumerate}
        \item If $\Gamma$ is a context and $\Gamma\proves M:\phi$, then
        $|\Gamma| \proves \phi$ in (intuitionistic) propositional logic
        \footnote{See \autoref{intuitionistic-logic}}.
        \item If $\Gamma\proves \phi$ in IPC 
        ({\bf I}ntuitionistic {\bf P}ropositional {\bf C}alculus), then
        you can define a context $\Gamma'=\{(x_\theta):\theta\mid\theta\in\Gamma\}$
        and find an $M: \phi$ such that $\Gamma'\proves M:\phi$. 
    \end{enumerate}
\end{theorem}

The previous discussion is mainly for application of functions, 
i.e. the {\it implicational fragment} of propositional logic.
Actually we only defined rules for how a `function type' $\sigma\to\tau$
is constructed by some $\lambda$-term and how to compute it by application.

We adopt the same style to define types for conjunction and disjunction.
The type set $\Delta$ is extended to
$$
\Delta ::= V\mid\Delta\to\Delta\mid\Delta\times\Delta\mid\Delta+\Delta
$$
with following annotating rules:
$$
\infer{\Gamma\proves (M, N): \sigma\times\tau}{\Gamma\proves M:\sigma, N:\tau}
\ \ \ 
\infer{\Gamma\proves \pi_1(L):\sigma}{\Gamma\proves L: \sigma\times\tau}
\ \ \ 
\infer{\Gamma\proves \pi_2(L):\tau}{\Gamma\proves L: \sigma\times\tau}
$$
$$
\infer{\Gamma\proves \left<1, M\right>:\sigma + \tau}{\Gamma\proves M:\sigma}
\ \ \ 
\infer{\Gamma\proves \left<2, N\right>:\sigma + \tau}{\Gamma\proves N:\tau}
$$
$$
\infer{\Gamma\proves \text{\bf case}(L;x.M;y.N): \rho}{
    \Gamma\proves L:\sigma+\tau &
    \Gamma, x:\sigma\proves M:\rho &
    \Gamma, y:\tau\proves N:\rho
}
$$
where 
$$
\begin{array}{c}
    \pi_1((M_1, M2)) \to_\beta M_1 \\
    \pi_2((M_1, M2)) \to_\beta M_2 \\
    \\
    \text{\bf case}(\left<1, N\right>;x.K;y.L)\to_\beta K[x:=N] \\ 
    \text{\bf case}(\left<2, N\right>;x.K;y.L)\to_\beta L[y:=N]
\end{array}
$$.

Of course, you may want to extend the $\lambda$-term to have such
syntactic definitions, but actually they can be expressed by lambda
calculus directly \cite{Curry-Howard} (in an inconvenient way). 
I prefer to define them syntactically instead of using lambda terms.
(The technique used to define those $\lambda$-terms is called {\it CPS 
transformation}, which is another interesting theoretical thing.)

\begin{caveat}
    \label{constructor-destructor}
    The notations $\pi_i$ and $\text{\bf case}$ are defined syntactically
    just like the lambda abstraction and application. They can only be
    introduced or eliminated according to the rules above. You cannot
    understand them like tuples in set theory. Instead, a better
    understanding is that you actually defined such constructors 
    $(,)_{\sigma\times\tau}: \sigma\to\tau\to\sigma\times\tau$ (the comma
    in the tuple) and $inj^1_{\sigma\times\tau}: \sigma\to\sigma+\tau$, 
    $inj^2_{\sigma\times\tau}: \tau\to\sigma+\tau$ like you are allowed
    to write the lambda abstraction. Similarly, the application $(M\ N)$
    never requires that $M$ should be an abstraction, you just think it 
    as an `applicable' term (e.g, with type $\sigma\to\tau$). You don't 
    know whether it has the form $(\lambda x.\cdots)$. It's purely a
    syntactic symbol. The type $\sigma\times\tau$ is also a symbol. We
    do not assume any structure like $(M, N)\in M\times N$. To make
    use of it, we define the destructor $\pi_1:\sigma\times\tau\to\sigma$
    and $\pi_2:\sigma\times\tau\to\tau$ and the computation rule. That is
    when you apply a constructor to a destructor, what should happen.
    Like the application $(M\ N)$, if $M$ is an abstraction, then we can 
    perform the $\beta$-reduction. This will be called {\it induction 
    rule} later with a more general description. That is to define
    such a rule, we only need to define them on all constructors.
    When a destructor is applied to a constructor, we define what is a
    reduction; outherwise, they are pure symbols (like $(M\ N)$).
    (You may want to think of them as axioms. The powerset axiom allows 
    you to define a set from another and others define how to use it.)
\end{caveat}

The above is the counterpart of propositional logic, which we often
refer to as {\it simply typed $\lambda$-calculus}. When defining
$\lambda$-terms which represents some kind of computation or logic,
we do not define the computation process or correctness directly, but 
its structure or the evidence that it is sound. 

\section{The Usage of Meta Language}
Though we have two things, $\lambda$-terms and types, as in set theory,
we will call them both `type'. A term must be introduced with its type 
and the type itself can be considered as the element in a higher order 
type universe. However the relation `$:$', unlike `$\in$', is not a 
propositional relation. This relation is judged in our meta-language. 
(Think about the rules above. The `$:$' relation is not derived from a 
rule, but a syntactic constraint.) According to the Curry-Howard 
Correspondence, when a proposition $A$ is provable if and only if a term 
$a$ of this type is constructable ($A$ is inhabited by $a$). Thus the 
relation $a:A$ is precisely the relation $\proves$ in our meta language. 

Another thing is the equality. In first order logic, we have a distinct
symbol `$\doteq$' for the language, but it is interpreted as the identity
in the model, i.e. the meta-language. In type theory, they are called
propositional equality and judgemental equality. We may prove `$=$' as
a proposition (we can find a term of this type), while the $\alpha$-
conversion (renaming of a $\lambda$-term) shows another equality, i.e.
the judgemental equality, an equality prior to the existence of type
theory. This judgement or definitional equality is denoted by `$\equiv$',
or sometimes to emphasize it's a definition, I will use `$\defn$'. In 
the rest of this chapter and \autoref{HoTT}, `$=$' will always mean
propositional equality as a type, and `$\equiv$' is the normal equality
we use for mathematics (judgemental equality). In other chapters, I will 
still use `$=$'. 

Now the only things we have to handle is the relation `$:$' and the 
judgemental equality $\equiv$, and to define the {\it function type} 
`$\to$', we first define the syntax for the type set, i.e., if $A$ and 
$B$ are type, then so is $A\to B$. This is called the {\bf formation 
rule} for $\to$. Then we have to show when this type is {\it inhabited}, 
i.e. to define an element $f$ of this type $f:A\to B$, I have to use 
$\lambda$-abstraction. This extends the syntax of $\lambda$-term. 
Such a syntax is called the {\bf constructor} of a new type 
(Another example is the comma $(,): A\to B\to A\times B$. Though I write 
it as a function, it is not defined as a function, but a new syntax. 
The formation rule is like you can always form a new formula but you
have to first give how to prove such a new formula by constructors and
only use constructors to `prove' this formula. The abstraction can never 
be described by a simpler term. Actually you can even introduce more 
syntax like the abstraction, e.g. the $\lambda\mu$-calculus 
\cite{lambda-mu-calculus}.) We also define the {\bf computation rule}
($\to_\beta$) $(\lambda x. M)\ N \defn M[x:=M]$ showing the 
{\bf elimination rule} for the function type. And we define the 
$\alpha$-conversion to identify two $\lambda$-terms, which is called 
the {\bf uniqueness rule}.

\section{The Universe of Type}
\label{type-universe}
I've only defined the syntax for types so far, but you may want to
ask, what is a type or what is the `universe' for types, just like the 
`set' of all sets? As in set theory, if the universe includes itself then
we will encounter Russell's paradox \cite{the-paradox-of-trees-in-type-theory}.
Instead, we define a hierarchical universe.
$$
    \universe_0: \universe_1: \universe_2: \cdots.
$$
This means each universe can be referred to in the language under a higher
level universe. We also require that if $A: \universe_i$ then 
$A: \universe_{i+1}$, but it has the disadvantage that an element won't
have a unique type and if you `proved' a theorem in $\universe_i$,
then you have to prove it again in universe $\universe_{i+1}$ because
you have more elements though the proofs are the same\footnote{In Idris
and Agda, the universe can be written in a `polymorphic' way, i.e. they
use dependent function type (\autoref{dependent-function}) to avoid this problem, maybe
this is a good solution in engineering. I'm not sure whether this will break
the consistency of the logic, but this inconvenience won't be a problem
in our development of type theory}. 

One would like to omit the subscription $i$ thus ignoring the 
inconvenience to some extent. For example, $A:\universe$ should have been 
some $A:\universe_i$ and $\universe:\universe$ is 
$\universe_i:\universe_{i+1}$, but be carefule that 
$\lambda(i:\nat). \universe_i$ is not a well defined term (The type 
$\nat$ will be introduced later, and $U_i$ is also understand as term.
There's no difference between a type and an element of that type. They
are all called {\it type}\footnote{In contrast to the traditional
programming language where {\it type} and {\it data} are different concepts}).

We need the universe $\universe$ to index a family of types, e.g. 
$B:A\to\universe$. A typical case is that when this $B$ is a `constant'
function. Let me also write that constant as $B:\universe$. Thus we can define
the constant function $(\lambda(x:A).B):A\to \universe$.


\section{Dependent Function Type}
The type family, as you may guess, is used to define some {\it polymorphic}
functions. For example a function of type $(A\to(B\to C))\to(A\to B)\to
(A\to C)$ can be define for every type $A, B, C$. You may expect such
a function with quantifier: $\forall (A\ B\ C: \universe), 
(A\to(B\to C))\to(A\to B)\to (A\to C)$, and define the term
$\lambda (A:\universe).\lambda (B:\universe).\lambda (C:\universe).\lambda (f: A\to B\to C).
\lambda (g: A\to B).\lambda (a: A). f\ a (g\ a): \forall (A\ B\ C:\universe), 
(A\to(B\to C))\to(A\to B)\to (A\to C)$.

Look at the term I defined above, it's too long to be readable and apparently, 
this is unnecessary. Once we give the type $\forall (A\ B\ C:\universe), 
(A\to(B\to C))\to(A\to B)\to (A\to C)$, we can infer the type of each 
variable in $\lambda A.\lambda B. \lambda C. \lambda (f: A\to B\to C). 
\lambda (B: A\to B).\lambda a. f\ a\ (g\ a)$ (though I add the redundant
types to $f$ and $g$ as hints). Actually there are two type systems, the
one I introduced is called {\it typing \`a la Curry}, which is a process
giving types to $\lambda$-terms. In contrast, there's another {\it typing
\`a la Church} which forces every variable to be assigned with a type.
Thue Curry typing is helpful when you want to write less type notations 
while the Church typing is how we write a proposition. There are some
further similarity and difference between these two type systems. For
more details, see Chapter 3 of \cite{Curry-Howard}. Actually I was
cheating because the Curry-Howard Correspondence should have been stated
with typing \`a la Church, but as you see, we don't have to be that strict
with ourselves. Generally I will write the type explicitly, but if it can
be inferred from the context, why not just omit them? The quantifiers also
gives us a way to get rid of `free variables', and we don't need to be that
strict. I will perhaps use subscriptions to indicate the bounded variables.
(For example, $f_{ABC}: (A\to B\to C)\to(A\to B)\to(A\to C)$).

In type theory, the `universal quantifiers' are represented by {\it
dependent function type ($\Pi$-types)}.

\begin{definition}
    \label{dependent-function}
    (Dependent function Type) Given a type $A:\universe$ and a family 
    $B:A\to\universe$, we form a new type $\Pi_{(x:A)}B(x): \universe$.

    The constructor for this type is still the lambda abstraction,
    but it allows you to introduce a type as a variable after the $\lambda$
    notation and that can be used in type annotation of a varaible later.

\end{definition}

For example the term $\lambda (A:\universe).\lambda (B:\universe).\lambda (C:\universe).\lambda (f: A\to B\to C).
\lambda (g: A\to B).\lambda (a: A). f\ a (g\ a)$ is such a constructor.
The dependent function type is a generalization of functions, for if
$B:A\to\universe$ is a constant function, the type $\Pi_{(x:A)}B$ is
the ordinary function $A\to B$. Sometimes the $\Pi$-type is called 
dependent function or dependent type. The name $\Pi$ is due to the fact
that dependent function can give a way to define product (exercise in
chapter 1 of \cite{homotopy-type-theory}).

Let me show some other examples.

\newcommand{\swap}{\name{swap}}

$$
    \swap: \prod_{A:\universe}\prod_{B:\universe}\prod_{C:\universe}
    (A\to B\to C)\to (B\to A\to C)
$$
$$
    \swap \defn \lambda A.\lambda B.\lambda C.\lambda g.
    \lambda b.\lambda a. g\ a\ b
$$

It is not wise to write so many $\lambda$s. Sometimes I accept the
traditional manner to define a function, i.e. I define the `{\bf swap}' as
$$
    \swap\ A\ B\ C\ g \defn \lambda b.\lambda a.g\ a\ b.
$$

This is understand as given the type $A,B,C$ and a function $g:A\to B\to C$,
{\bf swap} gives you a new function, which takes two arguments $b,a$ and gives
you the result $g\ a\ b$. Since the space is not `\LaTeX-friendly', I may still
use the notation 
$$
\swap(A,B,C,g)\defn \lambda b.\lambda a.g(a)(b).
$$
Be sure to distinguish this from tuple. (Especially when it comes to $f\ (a, b)$),
the application of a function to a tuple. You may also want to express this
as $(f\ (a, b)),f((a,b)),f(a,b)$ or the curried function $f(a)(b)$).

You can also use subscriptions to define it
$$
    \swap_{A,B,C}(g)(b,a) \defn g(a, b).
$$

The identity function is defined as
\newcommand{\id}{\name{id}}
\eq{
    \id &: \prod_{A:U} A\to A \\
    \id_A(x) &\defn x
}

\newcommand{\Id}{\name{Id}}
(This is the identity function, while later the type $\Id_A(a, b)$ 
will mean the equality of two elements $a,b:A$). 

\section{Inductively Defined Natural Numbers}
This time, we define a type with multiple constructors and induction.
Actually there should be some extra requirements\footnote{Inductive types
are initial algebras. In programming language Haskell, they are called
Algebraic data type} for induction because you may easily define the 
Rousell's paradox or some similar one (See chapter 5 of 
\cite{homotopy-type-theory}). 

The natural numbers $\nat$ is defined as a brand new type with
constructors $0: \nat$ and a successor function \newcommand{\succn}{\name{succ}} 
$\name{succ}:\nat\to\nat$. We also adopt the decimal notation 
$1 \defn \succn(0), 2\defn\succn(1)$. Though I write the $\succn$ like a
function, it behaves a bit different from a normal function. For example,
if $\succn(a) = \succn(b)$, we may then assume $a = b$ (equality is introduced
later), and we have to show how to `make use of' an element $n:\nat$.
To construct a function $f:\nat\to C$, I have to define what happens for
constructor $0: \nat$ and when $f(n)$ is defined, how to define $f(\succn(n))$.
Let $c_0:C$ and $c_s:\nat\to C\to C$ be two functions, we actually want
the (computation) rules

\eq{
    f(0) & \defn c_0 \\
    f(\succn(n)) &\defn c_s(n,f(n)).
}

In computability theory (recursion theory), this way to define a function
is called {\it primitive recursion}. 

\newcommand{\rec}{\name{rec}}

Though I give the universe of types, I still want to handle this pure 
syntactically. Instead of thinking $\nat$ consists of only $0$ and $\succn$,
this definition rule is written as the following recursion rule (the axioms
are symbolic).
$$
    \rec_\nat:\prod_{C: U}C\to(\nat\to C\to C)\to\nat\to C
$$
with computation rules
\eq{
    \rec_\nat(C, c_0,c_s,0) &\defn c_0 \\
    \rec_\nat(C, c_0,c_s,succ(n)) &\defn c_s(n, rec_\nat(C,c_0,c_s,n))
}
For example we can define the sum of two natural numberes;
$$
    \name{add}: \nat\to\nat\to\nat
$$
$$
    \name{add} \defn \rec_\nat(\nat\to\nat,\id_\nat,\lambda n.\lambda g.\lambda m.\succn(g(m)))
$$
This is, $\name{add}$ is a function from $\nat$ to $\nat\to\nat$ (when 
$C$ is $\nat\to\nat$). To define such a function, we first define 
$\name{add}\ 0$, which is apparently the identity function 
($0+n \defn n$), and if $\name{add}\ n$ is defined, we define 
$\name{add}\ (\succn\ n)$ to be the function which takes a number, applies
the previous $\name{add}\ n$ and takes the successor of the result, i.e.
a function $c_s(n:\nat, g:\nat\to\nat) \defn (\lambda m.\succn(g(m)): \nat\to\nat)$
($\name{add}(\succn(n))\defn \lambda m. \succn(add(n, m))$, i.e. 
$\name{add}(\succn(n), m)\defn \succn(add(n, m))$). (It's convenient
to write $\name{add}(n, m)$ as $n+m$)

Similarly we have 
$$
    \name{double}: \nat\to\nat
$$
$$
    \name{double} \defn \rec_\nat(\nat, 0, \lambda n.\lambda y.\succn(\succn(y))),
$$
and the predecessor
\newcommand{\pred}{\name{pred}}
$$
    \pred: \nat\to\nat
$$
$$
    \pred \defn \rec_\nat(\nat, 0, \lambda n.\lambda m. n),
$$
where $\pred(0)\defn0$ and $\pred(\succn(n))\defn (\lambda n.\lambda m. n)\ n\defn n$.

In the previous examples, you may notice that it suffices to define the
so called `computation rule' on each constructor. Actually there's a
{\it pattern matching} manner to define a recursive function, e.g. the
$\pred$ can be defined as
\eq {
    \pred(0) &\defn 0 \\
    \pred(\succn(n)) &\defn n
}
and the $\name{add}$ is defined as
\eq {
    0 + m &\defn m \\
    (\succn\ n) + m &\defn \succn\ (n + m).
}

But be sure such a definition can be written with $\rec$, or you may
get some pathological thing, e.g.
\eq{
    \name{loop\_false} &: \nat\to C \\
    \name{loop\_false}(n) &\defn \name{loop\_false}(n)
}
where $C:\universe$ is an arbitrary type. A system where every term
can be derived is inconsistent. Later I will construct a type $\zero$ 
without constructors. When $C$ is $\zero$, $\name{loop\_false}(0):\zero$ 
means absurdity (because no constructor is provided). So far we haven't
defined any `partial functions'. A turing machine may never {\it halt}
and $\name{loop\_false}$ is such a case. (This could be understood as a
computational version of Godel's incompleteness theory.)

Let's generalize this rule $\rec$ a bit. Suppose now I'm going to define
a dependent function on $\nat$ or predicate something about natural numbers,
i.e. you have the type $C: \nat\to\universe$. This is easily understand as
a predicate and to prove something about natural numbers, we can use inductions:
$$
    \infer{\Pi_{n:\nat}C(n)}{C(0) & \Pi_{n:\nat}(C(n)\to C(\succn(n)))}
$$
\newcommand{\ind}{\name{ind}}
Thus we have an induction rule 
$$
\ind_\nat: \prod_{C:\nat\to\universe}C(0)
    \to(\Pi_{n:\nat}C(n)\to C(\succn(n)))
    \to\Pi_{n:\nat} C(n).
$$

This is a generalized computation or elimination rule because when
$C$ is a constant type, it is just the recursion rule $\rec_\nat$. 
Thus we also have the following computation rule.
\eq{
    \ind_\nat(C,c_0,c_s,0) &\defn c_0 \\
    \ind_\nat(C,c_0,c_s,\succn(n)) &\defn c_s(n,\ind_\nat(C,c_0,c_s,n))
}
From our definition, this should be the only way to make use of a type.
See chapter 5 of \cite{homotopy-type-theory}. 

\newcommand{\refl}{\name{refl}}
Let me show you how to make use of this rule. There's a obvious `proposition'
$\forall n:\nat, n + 0 = n$. (Note from the computation rule, we only know
$0+n=n$.) The $=$ type is not introduced yet but its constructor is quite 
simple. Given a type $A:\universe$ and $x,y:A$, you can {\it form} a new 
type $x=_Ay$ while the only constructor for this type is $\refl_x: x=_Ax$. 
(Note that syntactically the {\it formation rule} always allows you to 
give a new type while the relation `$:$' is defined only by constructors.)

To prove that proposition, I expect the type $\Pi_{(n:\nat)}n+0 =_\nat n$
to be inhabited. According to the induction rule, I choose $C: \nat\to\universe$
to be $C(n)\defn n+0=_\nat n$. 
Now, I only need to `prove' $C(0)$ and $\Pi_{n:\nat}C(n)\to C(\succn(n))$.

$$
    C(0) \defn 0 + 0 =_\nat 0
$$
According to the computation rule, we know $0 + 0\defn 0$. Thus
$\refl_0: C(0)$ will be the witness. Now for the `induction step',
I reform the target as
$$
\infer{\succn(n) + 0 =_\nat \succn(n)}{n:\nat & n+0=_\nat n}
$$
Again by the computation rule, 
$\succn(n) + 0 =_\nat \succn(n)\defn \succn(n+0)=_\nat\succn(n)$. 
Now let's admit the leibniz rules. (though I am reluctant to do this, 
but it seems that without equality I could hardly prove anything. 
This will be introduced later.)

\begin{definition}
    \label{leibniz}
    $$
    \name{leibniz}: \prod_{A,B:\universe}\prod_{f:A\to B}\prod_{x,y:A}
        x=_Ay\to f(x)=_Bf(y)
    $$ 
\end{definition}
Clearly the type $\Pi_{n:\nat}C(n)\to C(\succn(n))$ can be inhabited by
$$
    \lambda (n:\nat).\lambda (eq:n+0=_\nat n). \name{leibniz}
    (\nat,\nat,\succn,n+0,n,eq). 
$$
This example also shows the difference between `judgemental equality' 
and `propositional equality'. The symbol $n+0$ will never equal $n$
judgementally, but they can be proved equal. 

\section{Some Simple Types}
Now that we have seen how a complicated type is defined (actually we
extend the typed lambda calculus), let me introduce some simpler types.
\newcommand{\inl}{\name{inl}}
\newcommand{\inr}{\name{inr}}
\begin{enumerate}
    \item The {\it product type} is defined with constructor {\it comma}
    $(,)_{A,B}:A\to B\to A\times B$. Actually if we have $a: A,b: B$, then
    we write $(a, b): A\times B$ omitting a lot of extra syntactic requirements.
    The {\it recursor} and {\it induction principle} are
    $$
    \rec_{A\times B}: \prod_{C:\universe}(A\to B\to C)\to A\times B\to C \\
    $$
    $$
    \rec_{A\times B}(C,g,(a,b)) \defn g(a)(b)
    $$
    $$
    \ind_{A\times B}:\prod_{C:A\times B\to\universe}
        (\prod_{x:A}\prod_{y:B}C((x,y)))\to\prod_{x:A\times B}C(x)
    $$
    $$
        \ind_{A\times B}(C,g(a,b))\defn g(a)(b)
    $$
    Here I defined the recursor and induction rules, instead of the
    equivalent $\pi_1,\pi_2$. They are defined as 
    $\pi_1\defn\rec_{A\times B}(A,\lambda a.\lambda b. a)$
    $\pi_2\defn\rec_{A\times B}(A,\lambda a.\lambda b. b)$.
    \item The unit type $\one$ is defined with only one constructor without
    any extra parameters ($\star:\one$). The recursor and induction principle
    are 
    $$
        \rec_\one:\prod_{C:\universe}C\to\one\to C
    $$
    $$ 
        \rec_\one(C,c,\star) \defn c
    $$
    $$
        \ind_\one:\prod_{C:\one\to\universe}C(\ast)\to\pi_{x:\one}C(x)
    $$
    $$
        \ind_\one(C,c,\ast) \defn c
    $$

    \item Given $A, B:U$, we form the {\it coproduct type} $A+B$ with
    constructors $\inl:A\to A+B$ $\inr:B\to A+B$ (`left injection' and
    `right injection'). The induction principle is
    $$
        \ind_{A+B}: \prod_{C:(A+B)\to\universe}
            (\prod_{a:A}C(\inl(a))) \to
            (\prod_{b:B}C(\inr(b))) \to
            \prod_{x:A+B} C(x)
    $$
    $$
        \ind_{A+B}(C,g_0,g_1,\inl(a))\defn g_0(a)
    $$
    $$
        \ind_{A+B}(C,g_0,g_1,\inr(a))\defn g_1(a)
    $$
    \item The type $\one$ is the nullary-product (terminal object). We 
    can similarly define the nullary-coproduct $\zero:\universe$. There's
    no way to construct the inhabitant of such a type, and we can easily
    figure out the induction principle 
    $\ind_\zero: \prod_{C:\zero\to\universe}\prod_{z:\zero}C(z)$. The
    simpler version $\rec_\zero:\prod_{C:\universe}\zero\to C$ implies
    that if $\zero$ had been inhabited, then any type, including itself, 
    would have been inhabited, i.e., any `proposition' would have been 
    provable. We then consider $\zero$ as absurdity $\bot$ in intuitionistic
    logic and I can give the counterpart of negation in Curry-Howard 
    Correspondence that $\neg\ P$ is understood as a function $P\to\zero$.
    To prove something is wrong, you prove that if this fact is true, 
    then you can prove absurdity and thus everything. The recursor 
    $\rec_\zero$ is the principle {\it ex falso quodlibet}.

    \item The {\it boolean} type $\two:\universe$ is a type with exactly
    two constructors $0_\two,1_\two:\two$. The induction principle is 
    $$
    \ind_\two:\prod_{C:\two\to\universe}C(0_\two)\to C(1_\two)\to\prod_{x:\two}C(x)
    $$
    $$
        \ind_2(C,c_0,c_1,0_\two)\defn c_0
    $$
    $$
        \ind_2(C,c_0,c_1,1_\two)\defn c_1
    $$
    Note this is not the boolean value of our language, but as a target.
    In first order logic, we define the {\it correctness} of a sentence
    by interpreting it to a boolean value (or more generally, a boolean
    algebra (Chapter 2 of \cite{Curry-Howard})), but this is not the
    (judgemental) correctness in our meta-language (like $\equiv,:$).
    For example, we can define a relation $\name{leb}:\nat\to\nat\to\two$ 
    (`less than or equal to in boolean') by pattern matching. (Please try
    rewriting it with the recursor $\rec_\nat$.)
    \eq {
        \name{leb}(0, m) & \defn 1_\two \\
        \name{leb}(\succn(n), 0) & \defn 0_\two \\
        \name{leb}(\succn(n),\succn(m)) & \defn \name{leb}(n,m)
    }
    However, we can also define this relation as a new type.
    Define $\name{le}: \nat\to\nat\to\universe$ (like the $\times:\universe
    \to\universe\to\universe$ for product type can also be understood as a 
    `function') with constructors $\name{le}_n(n): \name{le}(n, n)$ and
    $\name{le}_S(n:\nat, m:\nat, H: \name{le}(n, m)): \name{le}(n,\succn(m))$.
    (This is a dependent version of constructors. If you are not happy 
    with it, look at the next section.)
    You can try to prove\footnote{You can find this in the first volume of
    \cite{SF}. I strongly recommend you not prove it with your pen but with
    the help of a computer proof assistant.} 
    $\Pi_{n,m:\nat}\name{le}(n,m)\to \name{leb}(n,m)=_\two 1_\two$.
\end{enumerate}

\section{Dependent Pair}

Given type $A:\universe$, and a family $B:A\to\universe$, I want a relation
between some $a: A$ and $p: B(x)$. Trivially, let me just compose them
together with a tuple $(a, p)$. We call such a behavior {\it dependent 
pair} or $\Sigma$-type written as $(a,p):\Sigma_{x:A}B(x)$ because this 
looks like a product type while the type of the second element in the
tuple {\it dependent} on the first element. Of course, the constructor
is $(,)_{A,B:A\to\universe}: \Pi_{x:A}B(x)\to\Sigma_{y:A}B(y)$.
When $B$ is constant, $\Sigma_{x:A}B\defn(A\times B)$. Similarly we can
define $\pi_1:\Sigma_{x:A}B(x)\to A$ and 
$\pi_2:\Pi_{(p: \Sigma_{x:A}B(x))}B(\pi_1(p))$. Such a definition coincides
with the requirement for {\it existential quantifier} ($\exists$) in
intuitionistic logic, i.e. if something exists, you must construct it.
You can further make a group or at least a 
$\name{Magma}\defn \Sigma_{A:\universe}(A\to A\to A)$ with it.
The induction principle is
$$
    \ind_{\Sigma_{x:A}B(x)}:\prod_{C:(\Sigma_{x:A}B(x))\to\universe}
    (\Pi_{a:A}\Pi_{b:B(a)}C((a, b)))\to
    \prod_{p: \Sigma_{x:A}B(x)} C(p)
$$
$$
    \ind_{\Sigma_{x:A}B(x)}(C,g,(a,b)) \defn g(a)(b).
$$

The name $\Sigma$ is related to the disjoint unions. For example,
when $A\equiv \two$, we can define 
$A+B\defn \Sigma_{x:\two}\rec_\two(\universe,A,B,x)$ with 
$\inl(a)\defn(0_\two,a)$ and $\inr(b)\defn(1_\two,b)$. I'm not sure
whether this is true or not, but it seems that we can treat $\Pi$-types
and $\Sigma$-types as products and coproducts of indexed families of 
types. I haven't found any material about how to categorize dependent
types but as you may observe, there's a category of simply typed lambda
calculus. 

\section{Categroy of Simply Typed Lambda Calculus}
We use the definitional or judgemental equality $=_\beta$ as an
equivalence relation on the set of all $\lambda$-terms and make it
into a category:

\begin{itemize}[label={}]
    \item objects: the types
    \item arrows: If $c: A\to B$ is a $\lambda$-term without {\it free
    variables} (so-called closed term), then $[c]$ is an arrow from
    $A\to B$.
    \item Identities: $1_A$ is $[\lambda x: A.x]$.
    \item Compositions: If $[b]: A \to B$, $[c]: B\to C$, then
    $[c]\circ [b] = [\lambda x. c(b(x))]$.
\end{itemize}

\begin{remark}
    The equivalence relation defines a rule to `identity' different
    proofs. For example\footnote{This example comes from \cite{Curry-Howard}}
    $$
        \infer{\proves\lambda x:\phi.\lambda y:\phi.x:\phi\to\phi\to\phi}
        {
            \infer{x:\phi\proves\lambda y:\phi.x:\phi\to\phi}{
                x:\phi,y:\phi\proves x:\phi
            }
        }
    $$
    and
    $$
        \infer{\proves\lambda x:\phi.\lambda y:\phi.y:\phi\to\phi\to\phi}
        {
            \infer{x:\phi\proves\lambda y:\phi.y:\phi\to\phi}{
                x:\phi,y:\phi\proves y:\phi
            }
        }
    $$
    are two different proofs ($\lambda x. \lambda y. x\neq_\beta
    \lambda x.\lambda y. y$) of
    $$
        \infer{\proves\phi\to\phi\to\phi}{
            \infer{\phi\proves\phi\to\phi}{\phi\proves\phi}.
        }
    $$
\end{remark}

Such a category is a so-called CCC
({\it {\bf C}atesian {\bf C}losed {\bf C}ategory}).
The products and exponentials are obvious. Tranditionally we would
like to extend the syntax of types with two more constants $\top$($\one$)
and $\bot$($\zero$). Thus we have terminals and initials in this category. 

Given a $\lambda$-calculus $\mathcal{L}$ (different variable sets)
and a CCC $\mathbf{C}$, we define a {\it model} of $\mathcal{L}$ in
$\mathbf{C}$ as an assigement
\newcommand{\assign}[1]{\llbracket#1\rrbracket}
\eq {
    X\ \text{type} &\mapsto \assign{X}\ \text{object} \\
    b: A\to B\ \text{term} &\mapsto \assign{b}: \assign{A}\to\assign{B}\ \text{arrow}
}
such that
$$
\mathcal{L}\proves [a]=_\beta[b]: A\to B\text{ implies }
\assign{a}=\assign{b}: \assign{A}\to\assign{B}
$$

In \cite{cat-awodey}, the author mentioned that since topoi are
CCCs, it's natural to describe them with type theory. Combining
this with the first-order language, the subobject classifier provides
a natural interpretatioon of higher-order logic by exponentials of
the classifier. This part is vaguely described and is picked from the
book \cite{Introduction-to-higher-order-categorical-logic}. I think 
later I will dive into that book to fully understand this. 

\chapter{Homotopy Type Theory}
\label{HoTT}

In this chapter I mainly describe the `$=$' type and how it raises
a homotopy theory about types. First let's recall some homotopy theory
in algebraic topology.

\section{Homotopies}

This section is about algebraic topology. All `maps' are assumed to be
continuous unless mentioned otherwise. 

\begin{definition}
    \label{path}
    A {\it path} in space $X$ between two points is a map $p: I\to X$.
    If $p_1,p_2$ are two paths such that $p_1(1)=p_2(0)$, then we
    define their concatenation $p_1\centerdot p_2: I\to X$
    $$
        (p_1\centerdot p_2)(t) = \begin{cases}
            p_1(2\cdot t) & t\in[0,\frac{1}{2}] \\
            p_2(2\cdot(t-\frac{1}{2})), t\in[\frac{1}{2}, 1]
        \end{cases}
    $$
\end{definition}

\begin{definition}
    \label{homotopy}
    Given topological spaces $X,Y$, and $f,g:X\to Y$, a {\it homotopy} between
    $f$ and $g$ is a (continuous) map $H:X\times I\to Y$ such that
    \eq{
        H(x,0) &= f(x) \\
        H(x,1) &= g(x)
    }
    If $A\subseteq X$ is a subset and $f|_A=g|_A$, we can further require
    that $H(x, t)=f(x)=g(x)$ for $x\in A,t\in I$. Such a homotopy is called
    {\it relative homotopy} (homotopy relative to $A$). 
\end{definition}

We can use homotopy relative to the endpoints to define a equivalent
relation on paths. i.e. $[p_1]=[p_2]$ iff there exists a homotopy 
$H: I\times I\to X$ such that $H(0, t)=p_1(0)=p_2(0)$,$
H(1,t)=p_1(1)=p_2(1)$ and $H(s,0)=p_1(s)$, $H(t,1)=p_2(s)$.

\begin{definition}
    \label{fundamental-groupoid}
    Let $X$ be a topological space. The {\it fundamental groupoid\footnote{
        a groupoid is a category where each morphism has an inverse.
    }} $\pi X$ of it is
    a category where the objects are points in $X$ and morphisms
    between $x,y\in X$ is the equivalence classes $[p]$ of all paths
    $p$ between $x$ and $y$.
    The composition of $[p_1]: x\to y$ and $[p_2]:y\to z$ is
    $[p_2]\circ[p_1]=[p_1\centerdot p_2]$. (Note the different conventions
    of order for `composition in category' and `concatenation of paths'.)
    The identity at $x$ is the constant map $c_x$.
\end{definition}

This kind of definition ignores so-called `higher dimensional structures' 
or `higher order structures' because the morphisms between homotopies are
ignored. Traditionally one can define a homotopy between homotopies to
inspect those structures.

\begin{definition}
    If $f,g:X\to Y$ are maps and $F,G: X\times I\to Y$ are homotopies 
    between them, a homotopy between $F$ and $G$ is a map 
    $H: X\times I\times I\to Y$ such that
    $$
    \begin{array}{lcl}
        H(x,s,0) = F(x, s) & & H(x, s, 1) = G(x, s) \\
        H(x, 0, t) = f(x) & & H(x,1,t) = g(x).
    \end{array}
    $$
    This is to say $H$ is a homotopy between $F$ and $G$ and for each
    $t$, $H$ is still a homotopy between $f$ and $g$.
\end{definition}

If you believe a point in $X$ is a map $\one\to X$ and a path between
different points is a homotopy $\one\times I\to X$, then a homotopy
between two paths $p_1,p_2$ between $x,y$ relative to the endpoints
can be understood as a homotopy between homotopies.  
Now We can inspect the higher order homotopy structures by {\it track
groupoid}.

\begin{definition}
    For topological spaces $X$ and $Y$, the {\it track groupoid} $\pi Y^X$
    is a category with objects all continuous maps $X\to Y$ and morphisms
    the equivalence classes of all $homotopies$ between them. 
\end{definition}

The notation $Y^X$ is meaningful because we can equip the set $Y^X=\{
f: X\to Y\mid f\text{ continuous}\}$ with a {\it compact open}\footnote{
More generally, one can give the {\it test open} topology 
\cite{topology-and-groupoids}.} topology satisfying the exponential rule
$Hom(A\times X, Y)=Hom(A, Y^X)$ which is generated by the subbase
$\{V(K, U)\mid K\subseteq X\text{ compact},U\subseteq Y\text{ open}\}$,
where $V(K, U) = \{f: X\to Y\text{ continuous}\mid f(K)\subseteq U\}$.
Without surprise, The fundamental groupoid of the space $Y^X$ is the
track groupoid $\pi Y^X$. 

By this manner, we can definer higher order structures, namely the higher
order homotopy groupoids $\pi_n$. For example $\pi_0$ is the discrete
\footnote{A discrete category is a category where all morphims are
identities.} groupoid of all path components. But we can also define a
weak $\infty$-groupoid\footnote{I have been trying hard to understand
this concept, but it turns out that it requires lots of category theory
including (co)monad, (co)algebra and (co)induction 
\cite{types-are-weak-groupoids} and I've merely found
some fragmentary materials about this. I'm not sure whether
it's a good idea or not that I read the theory technically. Despite
this geometry confusion, I understand how a proof works in 
\cite{homotopy-type-theory}.} where we no longer take the equivalence 
classes and that category consists of points as the objects 
($0$-morphisms), paths as $1$-morphisms, homotopies as $2$-morphism, 
homotopies between homotopies as $3$-morphisms and so on, and the
equality required in identity rule or associativity may be replaced by 
homotopy $\simeq$ (higher order morphisms, namely the weakness).
\newcommand{\sphere}{\mathbb{S}}

The fundamental groupoid helps to classify topological spaces
geometrically. The homotopy intuitively means deformation retraction.
For example, let $\sphere^1=\{x\in\complex\mid |x| = 1\}$ with ordinary 
topology. Let $p(t) = e^{2\pi\iu t}: I\to \sphere^1$ be a path from
$1$ to $1$. There's no evidence that $[p]=[c_x]$ (the endpoints are 
fixed), but in the space ${\sphere^1}^I$ (the endpoints are free), 
they are equivalent (homotopic). Just like the algebraically topological
part of weak $\infty$-groupoid, this gives us an {\it intensional}
\footnote{In contrast to extensional type theory, another type theory
ignoring different possibilities of the equality.} interpretation of types.

\section{Identity Types}

As in the previous chapter, for a type $A:\universe$ and $x,y:A$,
we define a new type $x =_A y$. Without ambiguity, we can omit the $A$,
or write it as $\Id_A(x, y)$. The name `Identity' may cause confusion with
the identity function $\id_A$, so in Coq\cite{coq}, it is called {\it 
equality}. Now I follow the convention in \cite{homotopy-type-theory} still
calling it {\it identity type}.

\begin{definition}
    \label{identity-type}
    (Identity Type) The formation rule is 
    $\Id: \Pi_{A:\universe}A\to A\to\universe$ or $\Id_A: A\to A\to \universe$ 
    (note that $\to$ and $\times: \universe\to\universe$ are also 
    `formation rules') with an only constructor 
    $\refl:\Pi_{x:A}x=_A x$ or $\refl_x: x=_A x$ (one of the $x$ may
    be replaced by another judgementally equal term e.g. $0+0\equiv 0$).
\end{definition}

As usual, the induction principle is

\begin{equation}
    \label{ind_id}
    \ind_{=_A}:\prod_{(C:\prod_{(x,y:A)}(x=_A y)\to\universe)}
    (\prod_{x:A}C(x,x,\refl_x))\to\prod_{x,y:A}\prod_{p:x=_A y}C(x,y,p)
\end{equation}

with computation rule
$$
    \ind_{=_A}(C,c,x,x,\refl_x)\defn c(x).
$$
This rule, the {\it path induction}, is traditionally called $J$ and we 
can derive the {\it leibniz rule} (\autoref{leibniz}) from it. For 
convenience, we have another equivalent variation of it. (This variation 
is called {\it based path induction}.)
$$
    \ind'_{=_A}:\prod_{a:A}\prod_{(C:\Pi_{x:A}(a=_A x)\to \universe)}
    C(a, \refl_a) \to \prod_{x:A}\prod_{p:a=_A x}C(x, p)
$$

$$
    \ind'_{=_A}(a, C, c, a,\refl_a) \defn c.
$$

But be careful that we define the constructor for the space 
$\sum_{x,y:A}x=_A y$ (or $\sum_{y:A}(a=y)$ as for the based induction).
We may prove that $$(x,y,p)=_{\sum_{(x,y:A)}(x=y)}(x,x,\refl_x),$$ but
this does not mean that $\prod_{p:x=_A x}p=_{x=_Ax}\refl_x$ (because
we cannot even apply the induction rule), or more generally the {\it
Uniquess of Identity Proofs} (UIP).
\newcommand{\uip}{\name{uip}}
\begin{equation}
    \label{uip}
    \uip: \forall (A: \universe), (x,y:A), (p,q:x=_A y),p=_{x=_A y}q
\end{equation}
(It's quite silly to write a lot of $\Pi$s, especially when $\Pi$ exists
in the subscription, so I write $\forall$ instead.)
Note we that cannot define this function by pattern match 
$\uip(A,x,x,\refl,\refl\defn\refl$\footnote{In programming language
idris, this is acceptable.} because we potentially used an
equivalent $K$ axiom \cite{pattern-matching-without-K}
\begin{equation}
    \label{K-axiom}
    K: \forall (A:\universe), (a:A), (P: a=a\to\universe),
    P(\refl)\to\forall (l: a = a), P(l).
\end{equation}

It seems that there's a general way to give `sound' induction rules.
In fact, there's a $W$-type described in chapter 5 of \cite{homotopy-type-theory},
but I haven't reached that chapter. I know another similar application.
Recall the $\Ycomb$ combinator in \autoref{type-theory}, i.e. for any 
(untyped) $\lambda$-term $F$, $F\ (\Ycomb\ F)=_\beta \Ycomb\ F$.
Let $H=\lambda f.\lambda x.M$ and $F = \Ycomb\ H$. Then
\eq{
    F\ y & = (\Ycomb\ H)\ y \\
         & =_\beta H\ (Y\ H)\ y \\
         & = (\lambda f.\lambda x.M)\ (Y\ H)\ y \\
         & = (\lambda f.\lambda x.M)\ F\ y \\
         & = M[f:=F,x:=y].
}
$M[f:=F,x:=y]$ shows that the occurance of $f$ in $M$ will be the `defined'
(or not) $F$, i.e. this is a way to write recursive functions.
For example, if $M$ is something like $\mathbf{IF}\ C\ \mathbf{THEN}\ x
\ \mathbf{ELSE}\ f\ (\succn\ x)$, then $F\ c_n=_\beta M[f:=F,x:=c_n]=
\mathbf{IF}\ C\ \mathbf{THEN}\ c_n\ \mathbf{ELSE}\ F\ (\succn\ c_n)$.

\section{Homotopical Interpretation}
So how shall we understand the failure of $\uip$ rule? In 1996,
Martin Hofmann and Thomas Streicher proposed a homotopical model of
it \cite{the-groupoid-Interpretation-of-type-theory}. Instead of
defining a topological space and calculating its fundamental groupoid,
we define a homotopy theory synthetically, like the Euclidean geometry
versus analytic geometry based on $\mathbb{R}$. Given $p,q:a=_A a$,
there's no evidence that $p=_{a=_Aa}q$, which looks like the path
$t\mapsto e^{2\pi\iu t}$ is not homotopic to $c_1$. Each type $A$
is understood as a space with $p:a=_A b$ a path between two points in
this spaces. The type $a=_A b$ is thus the space of all such paths
from $a$ to $b$, so we can think of $H:p=_{a=_Ab}q$ as a homotopy in
higher order category. Just like the compact open topology, the space
$a=_Aa$ is different from the type $\sum_{y:A}a =_A y$, the space of
all paths from $a$. Thus we cannot assert $p=\refl_a$ and the judgemental
equality is different from this propositional equality. In this pattern,
the structure of the $\infty$-groupoid is given by defining the morphisms,
not from the topology and continuity (defining the hom-sets directly).

To call $p:a=_Ab$ a path, we have to first prove it has an `inverse' and
we can concatenate paths.

\begin{lemma}
    For $x,y:A$ and $p: x=y$, we have $p\inv: y=x$, i.e.
    a function $(\cdot)\inv: x=y\to y=x$ such that $\refl_x\inv=
    \refl_x$. $p\inv$ is the inverse of $p$.
\end{lemma}
\begin{proof}
    Let $D(x,y,p)\defn(y=x)$, $D:\prod_(x,y:A)(x=y)\to\universe$ and
    $d\defn(\lambda x.\refl_x):\prod D(x,x,\refl_x)$. Applying the
    induction principle \eqref{ind_id}, we get 
    $p\defn\ind_{=_A}(D,d,x,y,p):(y=x)$. According to the computation
    rule, $\refl_x\inv\defn \ind_{=_A}(D,d,x,x,\refl_x)\equiv d(x)\equiv
    \refl_x$.
\end{proof}
We can also make this proof shorter with {\it pattern matching}.
\begin{proof}
    By induction, it suffices to assume that $x\equiv y$ and $p$ is
    $\refl_x:x=x$. Clearly we can define $p\inv\defn \refl_x:x=x$, (which
    is exactly the computation rule).
\end{proof}

\newcommand{\sqcdot}{\mathbin{\vcenter{\hbox{\rule{.3ex}{.3ex}}}}}
\newcommand{\concat}{\sqcdot}
Similarly we can prove the following concatenation of paths.  
\begin{lemma}
    For $x,y,z: A$, $(\concat):(x=y)\to(y=z)\to(x=z)$ such that
    $\refl_x\concat \refl_x\equiv\refl_x$. 
\end{lemma}
\begin{proof}
    You can choose to define a $D(x,y,p)\defn\prod_{z:A}\prod_{q:y=z}(x=z)$
    and so on to invoke the induction principle \eqref{ind_id}, but it is
    a bit too tedious (since this time we need to invoke it twice).
    Instead, let's use pattern matching. By induction on $p$, it suffices 
    to assume $x\equiv y$ and $p\equiv\refl_x$. Now our target is
    $(x=x)\to(x=z)\to(x=z)$. Then let's continue by induction on $q$.
    Thus the target is $(x=x)\to(x=x)\to(x=x)$. We then define the last
    term to be $\refl_x$. 
\end{proof}
You may argue that why not apply id to $(x=x)\to(x=z)\to(x=z)$. The 
difference is that in the proof I presented, the computation rule is 
$\refl_x\concat\refl_x\equiv\refl_x$, while the other proof gives 
$\refl\concat q\equiv q$. Similarly you can define the computation
rule to be $p\concat\refl=p$. That is, as said in 
\cite{homotopy-type-theory}, only a stylishtic choice. The other two
should be derived (propositionally) from any of the three computation 
rules and we can have more results.
\begin{lemma}
    Suppose $x,y,z,w:A$ and $p:x=y$, $q:y=z$, $r:z=w$. Then
    \begin{enumerate}
        \item $p=p\concat\refl$, and $p=\refl\concat p$.
        \item $p\inv\concat p=\refl$, $p\concat p\inv=\refl$.
        \item ${(p\inv)}\inv=p$.
        \item $p\concat(q\concat r) = (p\concat q)\concat r$
    \end{enumerate}
\end{lemma}

Analogous to the fundamental group, we can define the {\it loop space} 
$$\Omega(A, a)\defn a=_Aa$$ for a type $A$ and $a: A$. More precisely,

\newcommand{\pointed}{_\bullet}
\begin{definition}
    We call $(A, a)$ a {\it pointed type} for $A:\universe$, $a:A$.
    $a$ is called the {\it basepoint} ane the pair of type 
    $\universe\pointed\defn\sum_{(A:\universe)A}$ is the type of pointed 
    types in universe $\universe$.
\end{definition}

\begin{definition}
    Given any pair $(A, a)$, we define the {\it loop space} as 
    $$ \Omega(A,a)\defn ((a=_A a),\refl_a).$$
    For $n:\nat$, we define the {\it$n$-fold iterated loop space} as
    $$\Omega^0(A,a)\defn(A,a)$$
    $$\Omega^{n+1}(A,a)\defn\Omega^n(\Omega(A, a))$$.
\end{definition}

For example, 
$$
    \Omega^2(A,a)\equiv\Omega(\Omega((A,a)))\equiv\Omega(a=_Aa,\refl_a)
    \equiv(\refl_a=_{a=_Aa}\refl_a, \refl_{\refl_a}).
$$

Just like ``{\it Groups in the category of groups are commutative}''
\cite{cat-awodey} or higher homotopy groups are commutative, we have
the Eckmann-Hilton theorem in type theory namely that the concatenation of 
paths $\Omega^2(A,a)\times\Omega^2(A,a)\to\Omega^2(A,a)$ is commutative.

The behavior of the type theory quite resembles homotopy in algebraic
topology. Let me give more notations.

In set theory, two functions $f,g: A\to B$ are equal iff 
$f(x)=g(x)\forall x\in A$ as ``relations''. In type theory this is wrong
because there's no evidence that $f: A\to B=g: A\to B$ ensures they are
defined with the same lambda term (with respect to $=_\beta$). For example 
$\lambda x:\nat. x+0$ and $\lambda x:\nat. x$ are definitionally different.
(note that $x+0=x$ is proved not $=\beta$). As proved before, we can
prove $\prod_{n:\nat}(\lambda x.x+0)\ n=(\lambda x.x)\ n$. You can even
verify this in pure $\lambda$-calculus (recall that natural numbers are
represented by church numerals). 
The axiom\footnote{This is not included by the theroy presented so far.}
$(\Pi_{x:A}f(x)=g(x))\to f = g$ is known as {\it function extensionality},
which will trivialize the intensional homotopy structure of the theory.
Thus in HoTT, this is viewed as a homotopy between $f$ and $g$:
\begin{definition}
    For $f,g:\Pi_{x:A}P(x)$, the homotopy relation between $f$ and $g$
    is a dependent function
    $$
        f\sim g\defn \prod_{x:A}f(x)=g(x)
    $$
    ($f$ is homotopic to $g$).
\end{definition}

\newcommand{\isequiv}{\name{isequiv}}
Similarly, we can define the homotopy equivalence $\simeq$.
\begin{definition}
    If $f:A\to B$ is a function, we say that $f$ is an equivalence provided
    $\exists g:B\to A,f\circ g\sim \id_B$ and 
    $\exists h:B\to A(h\circ f\sim \id_A)$, i.e., 
    $$
        \isequiv(f)\defn(\sum_{g:B\to A}f\circ g\sim \id_B)\times
        (\sum_{h:B\to A}h\circ f\sim \id_A)
    $$
    And we can define
    $$
        A\simeq B\defn\sum_{f:A\to B}\isequiv(f)
    $$
\end{definition}

Note we donnot define it as $\sum_{g:B\to A}(f\circ g\sim id_B)\times
(g\circ f\sim \id_A)$, where $g$ is called {\it quasi-inverse} of $f$.
One problem is that the inhabitants of that type is not unique 
(propositionally), but it can be shown that they are `equivalent' (two 
types imply each other), whose proof is similar to that in algebraic 
topology (a problem in the mid-term last semester). 

Now think about the path $=_A$ in the space $\universe$. We haven't 
inspected any further information about the universe $\universe$.
With the notation of equivalence, Voevodsky gave an extension of the
equality as an axiom (called {\it Univalence Axiom})
\newcommand{\univalence}{\name{univalence}}
$$
    \univalence: (A=B)\simeq(A\simeq B).
$$
You can easily find the canonical function $\Id_\universe(A,B)\to(A\simeq B)$. 
The univalence axiom extends it into an equivalence. It is said that this
axiom raises many consequences ``{\it naturally, simply and compellingly}''
\cite{homotopy-type-theory}. 

This homotopical interpretation is what I'm struggling with. I still
feel it a bit difficult for me to accept the unavailability of $\uip$
\eqref{uip} and I am trying to understand it as I understand `A implies B
is defined as not A or B'. Also, I don't feel quite happy with the
$\infty$-groupoid part, because I have never meet with such a higher
order category before. It seems that the book of HoTT\cite{homotopy-type-theory} 
wants to interpret the theory syntactically, because it does not require a 
profound comprehension about what is groupoid. Rather, it implies that the
homotopy type theory is a better language for this purpose. 

For example when defining a homotopical circle $\sphere$, we can define 
the it as a base point and a loop (path, i.e. the equality) between the 
basepoint
\newcommand{\base}{\name{base}}
\newcommand{\nloop}{\name{loop}}
$$
    \sphere\defn\begin{cases}
        \base: \sphere \\
        \nloop: \base=\base \\
    \end{cases}.
$$
This kind of style to define a new type (inductively) is called higher
inductive type (HIT) (chapter 6 of \cite{homotopy-type-theory}).
To calculate its fundamental group, we find the universal covering space 
of it, namely a family $\name{cov}:\sphere\to\universe$. That is to
find an $A: \universe$ and an equality $A=_\universe A$. Letting $A$
be $\integer$, we define $\name{cov}(\base)\defn\integer$ and 
$\name{conv}(\nloop):\base=\base$. According to the univalence axioms,
it suffices to define an equivalence $\integer\simeq\integer$. In fact
we know that $\succn: \integer\to\integer$ is clearly what we want.

The space $\sum_{x:\sphere}\name{cov}(x)$ forms a covering space of
the homotopical sphere $\sphere$. For any path $p:x=_\sphere y$ in A,
we have a {\it transport} $p_\ast:\name{cov}(x)\to \name{cov}(y)$ and
we can lift the path to $\sum_{x:\sphere}\name{cov}(x)$ with
\newcommand{\lift}{\name{lift}}
$\lift(u,p):(x,u)=(y,p_\ast(u))$, where $u: \name{cov}(x)$ is any
start point in the covering space. Thus we define the $\name{wind}$
function $\name{wind}: \Omega(\sphere,\base)\to\integer$ such that
for any $p:\base=\base$, $\name{wind}(p)=p_\ast(0)$, which is the
inverse of $n\mapsto \nloop^n: \integer\to\Omega(\sphere)$.

It's quite natural to define the circle $\sphere$ with this homotopy
interpretation. I think it's easier to describe the higher category,
e.g. the fundamental $\infty$-groupoid with the language of HoTT. 
Besides, it seems that this language has some other applications.
In talk given by Andr\'e Joyal at MIT \cite{joyal2014categorical}, 
it is suggested that the theory can be further applied to higher topos 
theory or higher category theory.


\chapter{Intuitionistic Logic}
\label{intuitionistic-logic}

\chapter{Monad}
\label{monad}

\chapter{Other Useful Examples}
\label{monad-examples}

\bibliographystyle{plain}
\bibliography{ref}

\end{document}
