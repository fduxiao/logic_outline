\documentclass[12pt,a4paper]{report}

\usepackage{amsmath,amssymb,amsthm}
\usepackage{titlesec}
\usepackage{hyperref}
\hypersetup{
    colorlinks=true,
    linktoc=all,
    linkcolor=black,
}

\renewcommand{\partname}{Part}
\renewcommand{\chaptername}{Chapter}

\begin{document}

\title{An Outline of Logic}
\author{Xiao}
\date{}
\maketitle

\tableofcontents

\chapter*{Introduction}
\addcontentsline{toc}{chapter}{Introduction} 

Several years ago, when I was programming, I was fascinated
by the concept of meta-programming, i.e., generating code with code
so that I could write less code. Generally, you write some code for
a real problem. Even if your target is to generate code, you have
to execute the code to get the result. This sounds nonsense to a 
mathematician, but this is actually how we handle mathematical 
logic, i.e., we define our language as mathematical objects 
(We define the well formed formula or proof as sets) 
just as we define a group with set theory. When we want to prove
something, we follow the syntax predefined (e.g. a deduction system).

one of my friends introduced Haskell to me.
I was excited by this appearently unconventional

In this outline, The first several chapters focus on a
formal language, namely the type theory and other 

\chapter{Type Theory}
\chapter{Intuitionistic Logic}
\chapter{Monad and Algebra}
\chapter{Combinatory Logic}
\chapter{Other Useful Examples}

\chapter{Homotopy Interpretation}

\chapter{aaa}
\section{Structure}
This section's content...

\subsection{Top Matter}
This subsection's content...

\subsubsection{Article Information}
This subsubsection's content...

For more infomation, read \cite[p.~215]{template}\cite{template}

\begin{equation}
    x = a_0 + \cfrac{1}{a_1 
            + \cfrac{1}{a_2 
            + \cfrac{1}{a_3 + \cfrac{1}{a_4} } } }
  \end{equation}


\begin{thebibliography}{}

\bibitem{Haskell}
    Haskell: \url{https://www.haskell.org}, 
    a functional programming language.

\bibitem{HoTT}
    The book,
    \textit{Homotopy Type Theory, Univalent Foundations of Mathematics},
    The Univalent Foundations Program, 
    Institute for Advanced Study,
    some edition,
    2020.


\bibitem{template}
    Author Somebody,
    \textit{\LaTeX: a document preparation system},
    publisher,
    some edition,
    2020.

\end{thebibliography}

\end{document}
