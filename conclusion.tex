\chapter*{Conclusion}
\addcontentsline{toc}{chapter}{Conclusion} 

The type theory gives another possibility to represent mathematics,
in such a way that the objects and the logics are represented by a
unified system of computation process ($\beta$-reduction $\to_\beta$). 
The identity between identites is not trivialized as in a fundamental
group. The higher homotopy relation between identites gives rise to
the homotopical interpretation of types. Together with Voevodsky's
univalence axiom, this is the homotopy type theory. It is at least a
good tool to describe geometry in a synthetical way. The expressibility
of such a logic is not weakened. The double negation translation
endorses classical logic in a monadic style. 

It can be shown that many mathematical objects or real world problems
can be represented in a monad. I wonder where does a monad attain its
power. The expressivity of it, I think, comes from the language, i.e.
a free construction, which is equivalent to an adjunction and therefore
a monad, an algebraic version to describe our language and our logic. 
There's a kind of combinatory logic and we can also define the CPS
transformation on it as some arbitrary logic relation control flow. 
The behavior is so widely applied in computer programming (though
it is not noticed by most people). It seems that they are strongly
connected. But in fact I seldom think a monad as an adjunction.
Instead I also define the Kleisli lift and use the monadic binding 
syntax. I think the lift itself has more abundant semantics. If you 
believe Curry-Howard correspondence, certainly we can develop mathematics 
like this. I hope this could help us to simplify objects, and thus find 
more structures. 
